\let\negmedspace\undefined
\let\negthickspace\undefined
\documentclass[journal]{IEEEtran}
\usepackage[a5paper, margin=10mm, onecolumn]{geometry}
%\usepackage{lmodern} % Ensure lmodern is loaded for pdflatex
\usepackage{tfrupee} % Include tfrupee package

\setlength{\headheight}{1cm} % Set the height of the header box
\setlength{\headsep}{0mm}     % Set the distance between the header box and the top of the text

\usepackage{gvv-book}
\usepackage{gvv}
\usepackage{cite}
\usepackage{amsmath,amssymb,amsfonts,amsthm}
\usepackage{algorithmic}
\usepackage{graphicx}
\usepackage{textcomp}
\usepackage{xcolor}
\usepackage{txfonts}
\usepackage{listings}
\usepackage{enumitem}
\usepackage{mathtools}
\usepackage{gensymb}
\usepackage{comment}
\usepackage[breaklinks=true]{hyperref}
\usepackage{tkz-euclide} 
\usepackage{listings}
% \usepackage{gvv}                                        
\def\inputGnumericTable{}                                 
\usepackage[latin1]{inputenc}                                
\usepackage{color}                                            
\usepackage{array}                                            
\usepackage{longtable}                                       
\usepackage{calc}                                             
\usepackage{multirow}                                         
\usepackage{hhline}                                           
\usepackage{ifthen}                                           
\usepackage{lscape}
\begin{document}

\bibliographystyle{IEEEtran}
\vspace{3cm}

\title{9.1.11}
\author{EE24BTECH11010 - Balaji B}
 \maketitle
% \newpage
% \bigskip
{\let\newpage\relax\maketitle}

\renewcommand{\thefigure}{\theenumi}
\renewcommand{\thetable}{\theenumi}
\setlength{\intextsep}{10pt} % Space between text and floats


\numberwithin{equation}{enumi}
\numberwithin{figure}{enumi}
\renewcommand{\thetable}{\theenumi}

\textbf{Question}:\\
Plot a solution to the following differential equation: $\brak{\frac{d^2y}{dx^2}}^3 + \brak{\frac{dy}{dx}}^2 + \sin{\brak{\frac{dy}{dx}}} + 1 = 0$
\\ \\
\solution
An exact theoretical solution using known methods of solving differential equations was not found; however, it can be approximated to a pretty good degree of precision. Euler's method will be used to obtain a plot of the solution\\
Computational Solution:\\
By first principle of derivatives,
\begin{align}
    y^{\prime}\brak{t} = \lim_{h\to 0}\frac{y\brak{t+h} - y\brak{t}}{h}\\
    y\brak{t+h} = y\brak{t} + hy^{\prime}\brak{t}
\end{align}
Let $y^{i}$ be the $i^{th}$ derivative of the function, $m$ be the order of the differential equation. Set $y_1=y, y_2=y^{1}, y_3=y^{2} \dots$, and so on.\\
We obtain the system,
\begin{align}
    \myvec{y_1^{\prime} \\ y_2^{\prime} \\ \vdots \\ y_{m-1}^{\prime} }&=\myvec{y_2 \\ y_3 \\ \vdots \\ y_m}\\
y_m^{\prime}&=f\brak{x, y_1, y_2, \dots, y_m}
\end{align}

Generalizing the system according to Euler's form\\
\begin{align}
  \myvec{y_1\brak{x+h} \\ \vdots \\ y_{m-1}\brak{x+h} \\ y_m\brak{x+h}} &=\myvec{y_1\brak{x} \\ \vdots \\ y_{m-1}\brak{x} \\ y_m\brak{x}} + h\myvec{y_2\brak{x} \\ \vdots \\ y_m\brak{x} \\f\brak{x, y_1, y_2, \dots, y_m} }
\end{align}
\begin{align}
  \vec{y}\brak{x+h} &= \vec{y}\brak{x} + h\myvec{0 & 1 & 0 & 0 & \dots & 0 & 0\\ 0 & 0 & 1 & 0 & \dots & 0 & 0\\0 & 0 & 0 & 1 & \dots & 0 & 0\\\vdots & \vdots & \vdots & \vdots& \ddots & \vdots & \vdots\\
  0 & 0 & 0 & 0 & \dots & 0 & 1\\0 & 0 & 0 & 0 & \dots & 0 &\frac{f\brak{x, y_1, y_2, \dots, y_m}}{y_m\brak{x}}}\vec{y}\brak{x}
  \end{align}
  \begin{align}
  \vec{y}\brak{x+h} &= \myvec{1 & h & 0 & 0 & \dots & 0 & 0\\ 0 & 1 & h & 0 & \dots & 0 & 0\\0 & 0 & 1 & h & \dots & 0 & 0\\\vdots & \vdots & \vdots & \vdots& \ddots & \vdots & \vdots\\
  0 & 0 & 0 & 0 & \dots & 1 & h\\0 & 0 & 0 & 0 & \dots & 0 & 1+\frac{f\brak{x, y_1, y_2, \dots, y_m}}{y_m\brak{x}}}\vec{y}\brak{x}
\end{align}
Discretizing the steps we get,
\begin{align}
  \vec{y}_{n+1}&=\myvec{1 & h & 0 & 0 & \dots & 0 & 0\\ 0 & 1 & h & 0 & \dots & 0 & 0\\0 & 0 & 1 & h & \dots & 0 & 0\\\vdots & \vdots & \vdots & \vdots& \ddots & \vdots & \vdots\\
  0 & 0 & 0 & 0 & \dots & 1 & h\\0 & 0 & 0 & 0 & \dots & 0 &1+\frac{f\brak{x, y_1, y_2, \dots, y_m}}{\brak{y_m}_n}}\vec{y}_{n}\\
    x_{n+1}&=x_n+h  
\end{align}
Where,
\begin{align}
  \vec{y}_n=\myvec{y_1\brak{x_n} \\ y_2\brak{x_n} \\ \vdots \\ y_m\brak{x_n}}
\end{align}
Smaller values of step size $h$ will give more precise plots. We obtain points to plot by iterating repeatedly. 
The given differential equation can be written as
\begin{align}
    \brak{\frac{d^2y}{dx^2}}^3 = -\brak{\frac{dy}{dx}}^2 - \sin{\brak{\frac{dy}{dx}}} - 1 
\end{align}
Here order $m$ is 2.
\begin{align}
  \vec{y}_{n+1}=\myvec{1 & h & 0 \\ 0 & 1 & h \\ 0 & 0 & 1+\frac{\brak{-\brak{\brak{y_1}_n^2 - \sin\brak{y_1}_n - 1 }}^{\frac{1}{3}} }{\brak{y_2}_n}}\vec{y}_{n}
\end{align}
Note, here the vector $\vec{y}$ is not to be confused with $y_i$ which represents a function, namely the ${i+1}^{th}$ derivative of $y\brak{x}$ Below is the plot for given curve  based on initial conditions, obtained by iterating through the above equation.

\begin{figure}[h!]
   \centering
   \includegraphics[width=1\columnwidth]{figs/fig1.png}
   \caption{Computational solution of $\brak{\frac{d^2y}{dx^2}}^3 + \brak{\frac{dy}{dx}}^2 + \sin{\brak{\frac{dy}{dx}}} + 1 = 0$}
   \label{stemplot}
\end{figure}
\end{document}
