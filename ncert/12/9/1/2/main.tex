\let\negmedspace\undefined
\let\negthickspace\undefined
\documentclass[journal]{IEEEtran}
\usepackage[a5paper, margin=10mm, onecolumn]{geometry}
%\usepackage{lmodern} % Ensure lmodern is loaded for pdflatex
\usepackage{tfrupee} % Include tfrupee package

\setlength{\headheight}{1cm} % Set the height of the header box
\setlength{\headsep}{0mm}     % Set the distance between the header box and the top of the text

\usepackage{gvv-book}
\usepackage{gvv}
\usepackage{cite}
\usepackage{amsmath,amssymb,amsfonts,amsthm}
\usepackage{algorithmic}
\usepackage{graphicx}
\usepackage{textcomp}
\usepackage{xcolor}
\usepackage{txfonts}
\usepackage{listings}
\usepackage{enumitem}
\usepackage{mathtools}
\usepackage{gensymb}
\usepackage{comment}
\usepackage[breaklinks=true]{hyperref}
\usepackage{tkz-euclide} 
\usepackage{listings}
% \usepackage{gvv}                                        
\def\inputGnumericTable{}                                 
\usepackage[latin1]{inputenc}                                
\usepackage{color}                                            
\usepackage{array}                                            
\usepackage{longtable}                                       
\usepackage{calc}                                             
\usepackage{multirow}                                         
\usepackage{hhline}                                           
\usepackage{ifthen}                                           
\usepackage{lscape}
\begin{document}

\bibliographystyle{IEEEtran}
\vspace{3cm}

\title{9.1.2}
\author{EE24BTECH11001 - Aditya Tripathy}
 \maketitle
% \newpage
% \bigskip
{\let\newpage\relax\maketitle}

\renewcommand{\thefigure}{\theenumi}
\renewcommand{\thetable}{\theenumi}
\setlength{\intextsep}{10pt} % Space between text and floats


\numberwithin{equation}{enumi}
\numberwithin{figure}{enumi}
\renewcommand{\thetable}{\theenumi}


\textbf{Question}:\\
Plot a solution to the following differential equation:\\
    $y^\prime + 5y = 0$
\\
\textbf{Solution: }\\
To plot a curve in the solution family, we take the initial condition to be\\
$x_0 = 0, y_0 = 1$\\
Using Euler's Method, we represent the the differential equation in the following difference equations:
\begin{align}
x_{n+1} = x_n + h\\
    y_{n+1} - y_n + 5hy_n  = 0 \xrightarrow{} y_{n+1} = y_n - 5hy_n
\end{align}
We can get the solution to the difference equation in $y$ using the unilateral Z-transform.
Applying the unilateral Z-transform to both sides of equation,$\brak{\text{representing transform of $y_n$ with $Y\brak{z}$}}$ 
\begin{align}
    Y\brak{z} = \sum_{n=0}^{\infty}y_nz^{-n}\\
\end{align}
    \text{Therefore,}\\
\begin{align}
    zY\brak{z} -zy_0 = \brak{1-5h}Y\brak{z}\\
    \xrightarrow{} Y\brak{z}\brak{z - \brak{1-5h}} = zy_0\\
    Y\brak{z} = \frac{zy_0}{z-\brak{1-5h}}\\
\end{align}
    \text{Applying Inverse Z-transform, we notice}\\
    \text{For the signal $x\sbrak{n} = \alpha^nu\brak{n}, \brak{u\sbrak{n} \text{being the unit step signal}}$}\\
\begin{align}
    X\brak{\alpha^nu\sbrak{n}} = \frac{1}{1-\alpha z^{-1}}, \text{provided $|z| < |\alpha|$}\\
\end{align}
    \text{On comparison, $\alpha = \brak{1-5h}$, so the solution for $y_n$ is:}\\
\begin{align}
    y_n = y_0\brak{1-5h}^nu\sbrak{n}\\
\end{align}
    \text{Substituting $y_0 = 1$,}\\
\begin{align}
    y_n = \brak{1-5h}^nu\sbrak{n}\\
\end{align}
Now we can iteratively generate points which lie close to the graph.\\
To check how close the approximate graph is to the actual solution, we will solve the original 
equation using a Laplace Transform method:\\
Let $\mathcal{L}\brak{y} = Y$\\
\begin{align}
    \brak{sY - y_0} +5Y = 0\\
    \brak{s + 5}Y = y_0\\
    Y = \frac{1}{s + 5}\\
\end{align}
    \text{Substituting $y_0 = 1$},\\
\begin{align}
    Y = \frac{1}{s + 5}\\
\end{align}
    \text{Now, we take the inverse laplace transform to get a solution,}\\
\begin{align}
    \mathcal{L}^{-1}\brak{\frac{1}{s+5}} = e^{-5x}u\brak{x}
\end{align}
For the following approximate graph, I chose $h = 0.01$ and $h = 0.1$.
\begin{figure}[h!]
   \centering
   \includegraphics[width=0.7\columnwidth]{figs/fig.png}
    \caption{Approximate solution of the DE}
\end{figure}
\end{document}  
