\let\negmedspace\undefined
\let\negthickspace\undefined
\documentclass[journal]{IEEEtran}
\usepackage[a5paper, margin=10mm, onecolumn]{geometry}
%
\setlength{\headheight}{1cm} % Set the height of the header box
\setlength{\headsep}{0mm}     % Set the distance between the header box and the top of the text

\usepackage{gvv-book}
\usepackage{gvv}
\usepackage{cite}
\usepackage{amsmath,amssymb,amsfonts,amsthm}
\usepackage{algorithmic}
\usepackage{graphicx}
\usepackage{textcomp}
\usepackage{xcolor}
\usepackage{txfonts}
\usepackage{listings}
\usepackage{enumitem}
\usepackage{mathtools}
\usepackage{gensymb}
\usepackage{comment}
\usepackage[breaklinks=true]{hyperref}
\usepackage{tkz-euclide} 
\usepackage{listings}
% \usepackage{gvv}                                        
\def\inputGnumericTable{}                                 
\usepackage[latin1]{inputenc}                                
\usepackage{color}                                            
\usepackage{array}                                            
\usepackage{longtable}                                       
\usepackage{calc}                                             
\usepackage{multirow}                                          
\usepackage{hhline}                                           
\usepackage{ifthen}                                           
\usepackage{lscape}
\begin{document}
	
	\bibliographystyle{IEEEtran}
	\vspace{3cm}
	\title{9.3.11}
	\author{EE24BTECH11022 - ESHAN SHARMA}
	% \maketitle
	% \newpage
	{\let\newpage\relax\maketitle}
	
	\renewcommand{\thefigure}{\theenumi}
	\renewcommand{\thetable}{\theenumi}
	\setlength{\intextsep}{10pt} % Space between text and floats
	
	
	\numberwithin{equation}{enumi}
	\numberwithin{figure}{enumi}
	\renewcommand{\thetable}{\theenumi}
	
	\textbf{Question:} Solve the differential equation A) \( \frac{d^2y}{dx^2} + y = 0 \) and verify if the general solution is $y= C_1e^{x} + C_2e^{-x}$.\\
	
	\solution
	
	\textbf{Solution Using Laplace Transform:}\\
	Given:
	\begin{align}
		\frac{d^2y}{dx^2} + y &= 0
	\end{align}
	Taking the Laplace Transform of both sides:
	\begin{align}
		\mathcal{L}\left\{\frac{d^2y}{dx^2}\right\} + \mathcal{L}\{y\} &= \mathcal{L}\{0\}
	\end{align}
	Using properties of the Laplace Transform:
	\begin{align}
		s^2Y(s) - sy(0) - y'(0) + Y(s) &= 0
	\end{align}
	Substituting the initial conditions \( y(0) = C_1 \) and \( y'(0) = C_2 \):
	\begin{align}
		s^2Y(s) - sC_1 - C_2 + Y(s) &= 0\\
		\left(s^2 + 1\right)Y(s) &= sC_1 + C_2\\
		Y(s) &= \frac{sC_1 + C_2}{s^2 + 1}
	\end{align}
	The Region of Convergence (ROC) is the entire \( s \)-plane since \( s^2 + 1 \neq 0 \) for all real \( s \).\\
	
	Taking the inverse Laplace Transform:
	\begin{align}
		y(x) &= \mathcal{L}^{-1}\left\{\frac{sC_1 + C_2}{s^2 + 1}\right\}\\
		&= C_1\cos(x) + C_2\sin(x)
	\end{align}
	Thus, the general solution is:
	\begin{align}
		y(x) = C_1\cos(x) + C_2\sin(x)
	\end{align}
	
	\textbf{Solving the Differential Equation using Z-Transform and Bilinear Transform:}\\
	Let the differential equation be:
	\[
	\frac{d^2y}{dx^2} + y = 0
	\]
	
	\textbf{Using the Z-Transform:}\\
	Taking the Z-transform of both sides:
	\begin{align}
		Z\left\{\frac{d^2y}{dx^2}\right\} + Z\{y\} &= Z\{0\} \\
		z^2 Y(z) - z y(0) - y'(0) + Y(z) &= 0
	\end{align}
	Rearranging terms:
	\begin{align}
		(z^2 + 1) Y(z) &= z y(0) + y'(0) \\
		Y(z) &= \frac{z y(0) + y'(0)}{z^2 + 1}
	\end{align}
	Substituting the initial conditions \(y(0) = y_0\) and \(y'(0) = y_1\), we get:
	\begin{align}
		Y(z) &= \frac{z y_0 + y_1}{z^2 + 1}
	\end{align}
	Taking the inverse Z-transform:
	\begin{align}
		y(x) &= \mathcal{Z}^{-1}\left( \frac{1}{z^2 + 1} \right) \\
		y(x) &= y_0 \cos(x) + y_1 \sin(x)
	\end{align}
	Thus, the solution using Z-transform is:
	\begin{align}
		y(x) = y_0 \cos(x) + y_1 \sin(x)
	\end{align}
	
	\textbf{Using the Bilinear Transform:}\\
	The Bilinear Transform is a method used to convert a continuous-time system (in the Laplace domain) into a discrete-time system (in the Z-domain). The mapping is given by the relation:
	\[
	s = \frac{2}{T} \cdot \frac{1 - z^{-1}}{1 + z^{-1}}
	\]
	where \( T \) is the sampling period, and \( z^{-1} \) is the inverse Z-transform variable.
	
	The continuous-time transfer function for the system is:
	\[
	H(s) = \frac{1}{s^2 + 1}
	\]
	
	\textbf{Step 1: Apply the Bilinear Transform}\\
	Substitute the Bilinear Transform relationship for \( s \) into the continuous-time transfer function \( H(s) \):
	\[
	H(z) = \frac{1}{\left( \frac{2}{T} \cdot \frac{1 - z^{-1}}{1 + z^{-1}} \right)^2 + 1}
	\]
	This equation expresses the continuous-time transfer function \( H(s) \) in terms of \( z \), mapping the system to the discrete-time domain.
	
	\textbf{Step 2: Deriving the Difference Equation}\\
	Simplify the denominator by expanding the square and combining terms:
	\[
	H(z) = \frac{(1 + z^{-1})^2}{(1 + z^{-1})^2 + \frac{4}{T^2}(1 - z^{-1})^2}
	\]
	Using partial fraction decomposition, we can determine the coefficients of \( H(z) \) and express it in terms of powers of \( z \). The resulting expression provides the difference equation:
	\[
	y[n+2] - 2y[n+1] + y[n] = 0
	\]
	where \( T \) determines the sampling rate.
	
	\textbf{Step 3: Verifying the Solution}\\
	The discrete-time system matches the sinusoidal behavior of the continuous-time system for sufficiently small \( T \), with the general solution:
	\[
	y[n] = C_1 \cos\left( \frac{\pi n}{T} \right) + C_2 \sin\left( \frac{\pi n}{T} \right)
	\]
	
	\textbf{Using Difference Equation to Approximate Solution:}\\
	This method approximates the solution by discretizing the function.\\
	From the definition of the second-order differentiation:
	\begin{align}
		\frac{d^2y}{dx^2} \approx \frac{y(x_{i+1}) - 2y(x_i) + y(x_{i-1})}{h^2}
	\end{align}
	Substituting into the differential equation:
	\begin{align}
		\frac{d^2y}{dx^2} + y = 0 \\
		\frac{y_{n+1} - 2y_n + y_{n-1}}{h^2} + y_n = 0
	\end{align}
	Simplifying:
	\begin{align}
		y_{n+1} = 2y_n - y_{n-1} - h^2y_n
	\end{align}
	Let \( x_0 = 0 \), \( y_0 = C_1 \), \( y_1 = C_1 + hC_2 \). For small step size \( h \), iteratively calculate \( y_{n+1} \).\\
	
	\begin{figure}[h]
		\centering
		\includegraphics[width=\textwidth]{figs/fig.png}
	\end{figure}
	
	By comparing the plots, the numerical solution matches the exact solution, verifying the correctness. Additionally, the plot of the given question does not match our solution, so Option A is not correct.
\end{document}
