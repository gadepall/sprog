\let\negmedspace\undefined
\let\negthickspace\undefined
\documentclass[journal]{IEEEtran}
\usepackage[a5paper, margin=10mm, onecolumn]{geometry}
\usepackage{lmodern} % Ensure lmodern is loaded for pdflatex
\usepackage{tfrupee} % Include tfrupee package

\setlength{\headheight}{1cm} % Set the height of the header box
\setlength{\headsep}{0mm}     % Set the distance between the header box and the top of the text

\usepackage{gvv-book}
\usepackage{gvv}
\usepackage{cite}
\usepackage{amsmath,amssymb,amsfonts,amsthm}
\usepackage{algorithmic}
\usepackage{graphicx}
\usepackage{textcomp}
\usepackage{xcolor}
\usepackage{txfonts}
\usepackage{listings}
\usepackage{enumitem}
\usepackage{mathtools}
\usepackage{gensymb}
\usepackage{comment}
\usepackage[breaklinks=true]{hyperref}
\usepackage{tkz-euclide} 
\usepackage{listings}                                      
\def\inputGnumericTable{}                                 
\usepackage[latin1]{inputenc}                                
\usepackage{color}                                            
\usepackage{array}                                            
\usepackage{longtable}
\usepackage{multicol}
\usepackage{calc}                                             
\usepackage{multirow}                                         
\usepackage{hhline}                                           
\usepackage{ifthen}                                           
\usepackage{lscape}
\begin{document}

\bibliographystyle{IEEEtran}
\vspace{3cm}

\title{12.6.5.6}
\author{EE24BTECH11006 - Arnav Mahishi}
% \maketitle
% \newpage
% \bigskip
{\let\newpage\relax\maketitle}

\renewcommand{\thefigure}{\theenumi}
\renewcommand{\thetable}{\theenumi}
\setlength{\intextsep}{10pt} % Space between text and floats


\numberwithin{equation}{enumi}
\numberwithin{figure}{enumi}
\renewcommand{\thetable}{\theenumi}


\textbf{Question}:\newline
Find the maximum profit that a company can make, if the profit function is given by $p\brak{x}=41-72x-18x^2$
\newline
\begin{table}[h!]    
  \centering
  \begin{tabular}{|c|c|}
    \hline
    \textbf{Variable} & \textbf{Description}\\
    \hline
    $P_0$ & initial principal amount\\
    \hline
    $r$ & rate of increase per year\\
    \hline
    $t$ & time in years\\
    \hline 
    $C \& C_1$ & arbitrary constants\\
    \hline
    $P$ & principal at any time $t$\\
    \hline
\end{tabular}

  \caption{Variables Used}
  \label{tab1.1.2.2}
\end{table}
\newline
\textbf{Theoretical Solution:}\\
To find critcal points we equate $\frac{dp\brak{x}}{dx}=0$. Let $y=p\brak{x}$
\begin{align}
    \frac{dy}{dx}=-72-36x\\
    \implies -72-36x=0\\
    \implies x=-2
\end{align}
To find whether $x=-2$ is a maxima or minima we need to take double derivative
\begin{align}
    \frac{d^2y}{dx^2}=-36
\end{align}
As the double derivative is negative for all $x$ so the point at $x=-2$ is a maxima\\
\begin{align}
    p\brak{2}=41-72\brak{-2}-18\brak{-2}^2=113
\end{align}
$\therefore$ The maximum profit the company can make is $113$ at $x=-2$\\
\textbf{Computational Solution:}\\
We use the method of gradient descent to find the local maximum of the given function. Since the coefficient of \brak{x^2 < 0}, the function is concave down, and we expect to find a local maximum. Hence we apply gradient ascent. The iterative formula \brak{\text{Difference Equation}} is as follows:
\begin{align}
    x_{n+1}&=x_n+\mu f^{\prime}\brak{x_n}\\
    f^\prime\brak{x_n}&=-72-36x_n\\
    \implies x_{n+1}&=x_n+\mu\brak{-72-36x_n}\\
    &=\brak{1-36\mu}x_n-72\mu
\end{align}
Applying Unilateral Z-transform,
\begin{align}
    zX\brak{z} - zx_0 &= \brak{1 - 36\mu}X\brak{z} - 72\mu \frac{z}{z-1} \\
    \brak{z - \brak{1 - 36\mu}}X\brak{z} &= zx_0 - 72\mu \frac{z}{z-1} \\
    X\brak{z} &= \frac{zx_0}{z - \brak{1 - 36\mu}} - \frac{72\mu z}{\brak{z-1}\brak{z - \brak{1 - 36\mu}}}
\end{align}
The ROC is determined by the stability condition:
\begin{align}
    |1 - 36\mu| &< 1 \\
    \implies -1 < 1 - 36\mu &< 1 \\
    \implies 0 < \mu &< \frac{1}{18}
\end{align}
If $\mu$ satisfies the previous condition,
\begin{align}
    \lim_{n\rightarrow\infty}\norm{x_{n+1}-x_n}=0\\
    \implies \lim_{n\rightarrow\infty}\norm{\mu\brak{-72-36x_n}}=0\\
    \implies -72\mu-36\mu\lim_{x\rightarrow\infty}\norm{x_n}=0\\
    \implies \lim_{x\rightarrow\infty}\norm{x_n}=-2
\end{align}
Choosing a step-size in the ROC $\brak{\mu=0.01}$, initial guess $x_0=0$ and tolerance $1\text{e-}5$, we perform $x_{n+1}=\brak{1-36\mu}x_n-72\mu$ until $f^\prime\brak{x}$ is less than the tolerance and we get $x_n$ to be the local maxima. After convergence we get:
\begin{align}
    x_{min}=-1.999997
\end{align}
We can pose the question as the following quadratic programming question. Find the point lying on the line $y=1$, which is nearest to the origin\\
We can formulate the problem as follows:
\begin{align}
    \min_{\vec{x}} \norm{e_2^{\Top}\vec{x}}^2\\
    \text{s.t. } \\ \vec{x}^{\Top}V\vec{x} + 2\vec{u}^{\Top}\vec{x} + f = 0\\
    V = \myvec{-18 & 0 \\ 0 & 0}\\
    \vec{u} = \myvec{-36 \\ -0.5}\\
    f = 0
\end{align}

In the current form, the constraint is non-convex since the constraint defines a set which is not convex, since points on the
line joining any 2 points on the curve don't belong to the set. However, if we become lenient and make the constraint
\begin{align}
    \vec{x}^{\Top}V\vec{x} + 2\vec{u}^{\Top}\vec{x}+f\le 0
\end{align}
The constraint becomes convex so we cant use cvxpy. Using scipy.optimize to solve this convex optimization problem, we get \\
\begin{align}
    \text{Optimal x}: [[-2.00000002]\\
 [113.000000006047071]]
\end{align}
\begin{figure}[h!]
   \centering
   \includegraphics[width=1\linewidth]{figs/fig.png}
   \caption{Maximum Value of Objective Function}
   \label{stemplot}
\end{figure}
\end{document}
