\let\negmedspace\undefined
\let\negthickspace\undefined
\documentclass[journal]{IEEEtran}
\usepackage[a5paper, margin=10mm, onecolumn]{geometry}
%\usepackage{lmodern} % Ensure lmodern is loaded for pdflatex
\usepackage{tfrupee} % Include tfrupee package

\setlength{\headheight}{1cm} % Set the height of the header box
\setlength{\headsep}{0mm}     % Set the distance between the header box and the top of the text

\usepackage{gvv-book}
\usepackage{gvv}
\usepackage{cite}
\usepackage{amsmath,amssymb,amsfonts,amsthm}
\usepackage{algorithmic}
\usepackage{graphicx}
\usepackage{textcomp}
\usepackage{xcolor}
\usepackage{txfonts}
\usepackage{listings}
\usepackage{enumitem}
\usepackage{mathtools}
\usepackage{gensymb}
\usepackage{comment}
\usepackage[breaklinks=true]{hyperref}
\usepackage{tkz-euclide} 
\usepackage{listings}
% \usepackage{gvv}                                        
\def\inputGnumericTable{}                                 
\usepackage[latin1]{inputenc}                                
\usepackage{color}                                            
\usepackage{array}                                            
\usepackage{longtable}                                       
\usepackage{calc}                                             
\usepackage{multirow}                                         
\usepackage{hhline}                                           
\usepackage{ifthen}                                           
\usepackage{lscape}
\begin{document}

\bibliographystyle{IEEEtran}
\vspace{3cm}

\title{8.2.6}
\author{EE24BTECH11010 - Balaji B}
 \maketitle
% \newpage
% \bigskip
{\let\newpage\relax\maketitle}

\renewcommand{\thefigure}{\theenumi}
\renewcommand{\thetable}{\theenumi}
\setlength{\intextsep}{10pt} % Space between text and floats


\numberwithin{equation}{enumi}
\numberwithin{figure}{enumi}
\renewcommand{\thetable}{\theenumi}

\textbf{Question}:\\
Smaller area enclosed by the circle $x^2 + y^2 = 4 $ and the line $x + y = 2 $ is 
\\ \\
\textbf{Solution:}\\
\begin{table}[H]
    \centering
    \begin{table}[h]
    \centering
    \renewcommand{\arraystretch}{1.3}
    \begin{tabular}{|c|c|c|}
        \hline
        \textbf{SEVENSEG} & \textbf{PINS} & \textbf{RESISTOR USAGE} \\
        \hline
        A& 2 & NO\\
        B& 3& NO\\
        C& 4& NO\\
        D& 5& NO\\
        E& 6& NO\\
        F& 7& NO\\
        G& 8& NO\\
        DOT& GND & NO\\
        \hline
    \end{tabular}
    \caption{Seven-Segment Display Pin Mapping and Resistor Usage}
    \label{tab:sevenseg}
\end{table}


    \caption{Variables used}
    \label{tab1-1.2-20}
\end{table} 
\textbf{Theoritical Solution: }

The point of intersection of the line with the circle is $x_i=h+k_i m$,\\
where, $k_i$ is a constant and is calculated as follows:-
$$k_i=\frac{1}{m^\top Vm}\brak{-m^\top \brak{Vh+u}\pm \sqrt{\sbrak{m^\top \brak{Vh+u}} 2-g\brak{h}\brak{m^\top Vm}}}$$\\
Substituting the input parameters into $k_i$,\\
\begin{multline}
     k_i =\frac{1}{\myvec{1 
 & -1}\myvec{1 & 0 \\ 0 & 1}\myvec{1 \\ -1}}\brak{-\myvec{1&-1}\brak{\myvec{1&0\\0&1}\myvec{2\\0}+\myvec{0\\0}}}\pm \\
     \sqrt{\sbrak{\myvec{1&-1}\brak{\myvec{1&0\\0&1}\myvec{2\\0}+\myvec{0\\0}}}^2-g\brak{h}\brak{\myvec{1&-1}\myvec{1&0\\0&1}\myvec{1\\-1}}} 
\end{multline}
We get,\\
$k_i= 0,-2$\\
Substituting $k_i$ into $x_i=h+k_i m$ we get\\
\begin{align}
     x_1&=\myvec{2\\0}+\brak{0}\myvec{1\\-1}\\
    \implies x_1 &=\myvec{2\\0}\\
    x_2 &=\myvec{2\\0}+\brak{-2}\myvec{1\\-1}\\
    \implies x_2&=\myvec{2\\0}+\myvec{-2\\2}\\
    \implies x_2&=\myvec{0\\2}
\end{align}
The area of the smaller region bounded by the circle $x^2 + y^2 = 4$ and the line $x + y = 2$ is
\begin{align}
    &=\int_{0}^{2} \sqrt{4-x^2}dx-\int_{0}^{2} \brak{2-x} dx\\
    &=\brak{\frac{x}{2}\sqrt{4-x^2}+\frac{4}{2}\sin^{-1}{\frac{x}{2}}-2x+\frac{x^2}{2}}_{0}^{2}\\
    &=\brak{\pi-2}
\end{align}

\textbf{Computational Solution:}\\

Taking trapezoid-shaped strips of small area and adding them all up. Say we have to find the area of $y_{x}$ from $x=x_0$ to $x=x_n$, discretize the points on the $x$ axis $x_0, x_1, x_2, \dots, x_n$ such that they are equally spaced with the step size $h$. \\
Sum of all trapezoidal areas is given by,
\begin{align}
  A&=\frac{1}{2}h\brak{y\brak{x_1}+y\brak{x_0}}+ \frac{1}{2}h\brak{y\brak{x_2}+y\brak{x_1}}+\dots+\frac{1}{2}h\brak{y\brak{x_n}+y\brak{x_{n-1}}}\\
  &=h\sbrak{\frac{1}{2}\brak{y\brak{x_0}+y\brak{x_n}}+ y\brak{x_1}+\dots+y\brak{x_{n-1}}}
\end{align}
Let $A\brak{x_n}$ be the area enclosed by the curve $y\brak{x}$ from $x=x_0$ to $x=x_n$, $\brak{x_0, x_1, \dots x_n}$ be equidistant points with step-size $h$.
\begin{align}
  A\brak{x_n+h}=A\brak{x_n}+\frac{1}{2}h\brak{y\brak{x_n+h}+y\brak{x_n}}
\end{align}
We can repeat this till we get required area.\\
Discretizing the steps, making $A\brak{x_n}=A_n, y\brak{x_n}=y_n$ we get,
\begin{align}
 A_{n+1}=A_n+\frac{1}{2}h\brak{y_{n+1}+y_n}
\end{align}
We can write $y_{n+1}$ in terms of $y_n$ using first principle of derivative. $y_{n+1}=y_n+hy^{\prime}_n$
\begin{align}
  A_{n+1}&=A_n+\frac{1}{2}h\brak{\brak{y_{n}+hy^{\prime}_n}+y_n}\\
  A_{n+1}&=A_n+\frac{1}{2}h\brak{2y_n+hy^{\prime}_n}\\
  A_{n+1}&=A_n+hy_n+\frac{1}{2}h^2y^{\prime}_n\\
  x_{n+1}&=x_n+h
\end{align}
In the given question, $y_n=\sqrt{4-x_n^2} + x_n -2$ and $y^{\prime}_n= \frac{-x_n}{\brak{\sqrt{4-x_n^2}}} +1 $\\
The general difference equation will be given by
\begin{align}
  A_{n+1}&=A_n+hy_n+\frac{1}{2}h^2y^{\prime}_n\\
  A_{n+1}&=A_n+h\brak{\sqrt{4-x_n^2} + x_n -2}+\frac{1}{2}h^2\brak{\frac{-x_n}{\brak{\sqrt{4-x_n^2}}} +1}\\
  x_{n+1}&=x_n+h
\end{align}
Iterating till we reach $x_n=2$ will return required area. \\
Area obtained computationally: $1.41555$ sq. units\\
Area obtained theoretically: $\brak{\pi - 2}$ sq. units = 1.14 sq.unis
\begin{figure}[h!]
   \centering
   \includegraphics[width=1\columnwidth]{figs/fig.png}
   \caption{Graph of the cirlce $x^2 + y^2 = 4$ and $x + y = 2 $ and the area enclosed between them}
   \label{stemplot}
\end{figure}
\end{document}
