\let\negmedspace\undefined
\let\negthickspace\undefined
\documentclass[article]{IEEEtran}
\usepackage[a5paper, margin=10mm, onecolumn]{geometry}
%\usepackage{lmodern} % Ensure lmodern is loaded for pdflatex
\usepackage{tfrupee} % Include tfrupee package

\setlength{\headheight}{1cm} % Set the height of the header box
\setlength{\headsep}{0mm}     % Set the distance between the header box and the top of the text

\usepackage{gvv-book}
\usepackage{gvv}
\usepackage{cite}
\usepackage{amsmath,amssymb,amsfonts,amsthm}
\usepackage{algorithmic}
\usepackage{graphicx}
\usepackage{textcomp}
\usepackage{xcolor}
\usepackage{txfonts}
\usepackage{listings}
\usepackage{enumitem}
\usepackage{mathtools}
\usepackage{gensymb}
\usepackage{comment}
\usepackage[breaklinks=true]{hyperref}
\usepackage{tkz-euclide} 
\usepackage{listings}                                       
\def\inputGnumericTable{}                                 
\usepackage[latin1]{inputenc}                                
\usepackage{color}                                            
\usepackage{array}                                            
\usepackage{longtable}                                       
\usepackage{calc}                                             
\usepackage{multirow}                                         
\usepackage{hhline}                                           
\usepackage{ifthen}                                           
\usepackage{lscape}

\renewcommand{\thefigure}{\theenumi}
\renewcommand{\thetable}{\theenumi}
\setlength{\intextsep}{10pt} % Space between text and floats

\numberwithin{figure}{enumi}
\renewcommand{\thetable}{\theenumi}

% Marks the beginning of the document
\begin{document}
\bibliographystyle{IEEEtran}
\title{NCERT-8.2.3}
\author{EE24BTECH11035 - KOTHAPALLI AKHIL}
{\let\newpage\relax\maketitle\vspace{-1.20cm}}
\noindent\textbf{Question: }  
Find the area of the region bounded by the curves $y = x^2 + 2$, $y = x$, $x = 0$, and $x = 3$.\\
\solution \\
\noindent\textbf{Finding the Points of Intersection:}\\
To find the points of intersection of the curves $y = x^2 + 2$ and $y = x$, we equate the two equations:
\begin{align}
x^2 + 2 &= x
\end{align}
Rearranging the terms:
\begin{align}
x^2 - x + 2 &= 0
\end{align}
This can be written in the standard quadratic form:
\begin{align}
ax^2 + bx + c &= 0
\end{align}
where $a = 1$, $b = -1$, and $c = 2$. Using the quadratic formula:
\begin{align}
x &= \frac{-b \pm \sqrt{b^2 - 4ac}}{2a}
\end{align}
Substituting the values of $a$, $b$, and $c$:
\begin{align}
x &= \frac{-(-1) \pm \sqrt{(-1)^2 - 4(1)(2)}}{2(1)}
\end{align}
Simplify:
\begin{align}
x &= \frac{1 \pm \sqrt{1 - 8}}{2}
\end{align}
\begin{align}
x &= \frac{1 \pm \sqrt{-7}}{2}
\end{align}
Since the discriminant $b^2 - 4ac = -7$ is negative, the equation has no real roots.\\

\noindent\textbf{Intersection with $x = 0$:}\\
For $x = 0$, substitute into the equations of the curves:
\begin{align}
y &= x^2 + 2 = 0^2 + 2 = 2 \\
y &= x = 0
\end{align}
Thus, the point of intersection with $x = 0$ is $(0, 2)$.\\

\noindent\textbf{Intersection with $x = 3$:}\\
For $x = 3$, substitute into the equations of the curves:
\begin{align}
y &= x^2 + 2 = 3^2 + 2 = 9 + 2 = 11 \\
y &= x = 3
\end{align}
Thus, the point of intersection with $x = 3$ is $(3, 11)$.\\

\noindent\textbf{Conclusion:}\\
The curves $y = x^2 + 2$ and $y = x$ do not intersect in the real domain. However, the intersections with the vertical lines $x = 0$ and $x = 3$ are $(0, 2)$ and $(3, 11)$, respectively. These points define the bounds for calculating the area of the region enclosed by the curves.

\textbf{Theoritical approach: }\\
The area of the region can be expressed as:
\begin{align}
   A = \int_0^3 \left( x - (x^2 + 2) \right) dx. 
\end{align}

Simplify the integrand:
\begin{align}
A = \int_0^3 \left( -x^2 + x - 2 \right) dx.
\end{align}
on integration,
\begin{align}
    A={\left[-\frac{x^3}{3}+\frac{x^2}{2}-2x\right]_0}^3
\end{align}
on applying the limits,
\begin{align}
   A=(-\frac{3^3}{3}+\frac{2^2}{2}-6)-(0+0-0)
\end{align}
\begin{align}
    A=10.50
\end{align}
Here, we are getting A negative .That means it is area under the X-axis.
Therefore,the maginitude of theoritical area is 10.50.\\
\textbf{Using the trapezoidal method,}\\ 
For finding the approximate area enclosed using iterative methods, we use the Trapezoidal method. We make the area into multiple small trapeziums, and we sum up all the trapezium areas to find the total area. \\
We divide the $x$-coordinates with uniform step-size $h \to 0$, such that the discretized points are $x_0$, $x_1$, $\dots$, $x_n$ and $x_{n + 1} = x_n + h$.
\newline
Let the sum of trapizoidal areas till $x_n$ be $A_n$ and $y = y\brak{x}$, then we write the $\textbf{Difference equation}$,
\begin{align}
    A_n &= \frac{h}{2}\brak{y\brak{x_0} + y\brak{x_1}} + \frac{h}{2}\brak{y\brak{x_1} + y\brak{x_2}} + \dots + \frac{h}{2}\brak{y\brak{x_{n - 1}} + y\brak{x_{n}}}\\
    A_n &= h\brak{\frac{y\brak{x_0}}{2} + y\brak{x_1} + y\brak{x_2} \dots \frac{y\brak{x_n}}{2}}\\
    A_{n + 1} &= A_n + \frac{h}{2}\brak{y\brak{x_{n + 1}} + y\brak{x_n}} \text{, } x_{n + 1} = x_n + h\\
    A_{n + 1} &= A_n + \frac{h}{2}\brak{y\brak{x_n + h} + y\brak{x_n}}\\
\end{align}
By the first principle of derivative,
\begin{align}
    y^{\prime}\brak{x} &= \lim_{h\to0} \frac{y\brak{x + h} - y\brak{x}}{h}\\
    y\brak{x + h} &= y\brak{x} + h\brak{y^{\prime}\brak{x}} \text{, } h\to0
\end{align}
Rewriting the difference equation, we get,
\begin{align}
    A_{n + 1} &= A_n + \frac{h}{2}\brak{y\brak{x_n} + hy^{\prime}\brak{x_n} + y\brak{x_n}}\\
    A_{n + 1} &= A_n + h\brak{y\brak{x_n} + \frac{h}{2}y^{\prime}\brak{x_n}}\\
    A_{n + 1} &= A_n + hy\brak{x_n} + \frac{h^2}{2}y^{\prime}\brak{x_n}
\end{align}\\
Divide the interval $[0, 3]$ into $n$ subintervals of width $h = \frac{3-0}{n} = \frac{3}{n}$. Let $x_0 = 0$, $x_1 = h$, $x_2 = 2h$, $\dots$, $x_n = 3$.

The trapezoidal method for numerical integration is given by:
\begin{align}
  A \approx \frac{h}{2} \left[ f(x_0) + 2\sum_{i=1}^{n-1} f(x_i) + f(x_n) \right]  
\end{align}

where $f(x) = -x^2 + x - 2$.

Substitute $f(x)$ into the formula:
\begin{align}
    A \approx \frac{h}{2} \left[ (-x_0^2 + x_0 - 2) + 2\sum_{i=1}^{n-1} (-x_i^2 + x_i - 2) + (-x_n^2 + x_n - 2) \right].
\end{align}
\textbf{Numerical Approach}\\

To approximate the area of the region using the trapezoidal method, we adopt the following numerical approach. This method involves iteratively calculating the area using discrete steps:

\textbf{Initialization:}
\begin{itemize}
    \item Start with initial values $x_0 = 0$ and $A_0 = 0$.
    \item Set the step size $h = \frac{3}{n}$, where $n$ is the number of subintervals.
\end{itemize}

\textbf{Iterative formulae:}\\
\begin{itemize}
    \item For each iteration $i$ from 1 to $n$, perform the following steps:
    \begin{itemize}
        \item[1.] Compute $y_i = -x_i^2 + x_i - 2$.
        \item[2.] Update the area sum using:
        \begin{align}
        A_{i} &= A_{i-1} + \frac{1}{2} h (y_{i} + y_{i-1})
        \end{align}
        \item[3.] Update $x_i$ for the next iteration using:
        \begin{align}
        x_{i} &= x_{i-1} + h
        \end{align}
    \end{itemize}
\end{itemize}

\textbf{Final Area Calculation:}\\
\begin{itemize}
    \item After completing all iterations, the final approximate area $A_n$ is:
    \begin{align}
    A &= A_n
    \end{align}
\end{itemize}

\textbf{Initial Conditions:}
\begin{itemize}
    \item $x_0 = 0$
    \item $A_0 = 0$
    \item $h = \frac{3}{n}$ (depending on the chosen number of subintervals $n$)
    \item Here we assume $n = 30$.
\end{itemize}

This approach ensures an accurate approximation of the area by iteratively applying the trapezoidal rule, leveraging the discretized nature of the integral.\\
$ \implies$ The theoritical value of Area is 10.50\\
 $ \implies$ The computational value of Area is 10.505.\\
 Therefore, we can claim that trapezoidal rule/method for finding area works well.
 \begin{figure}[h!]
	\centering
	\includegraphics[width=\columnwidth]{figures/Figure_1.png}
	\caption{Area function graph.}
	\label{stemplot}
\end{figure}
\end{document}



