\let\negmedspace\undefined
\let\negthickspace\undefined
\documentclass[journal]{IEEEtran}
\usepackage[a5paper, margin=10mm, onecolumn]{geometry}
%\usepackage{lmodern} % Ensure lmodern is loaded for pdflatex
\usepackage{tfrupee} % Include tfrupee package

\setlength{\headheight}{1cm} % Set the height of the header box
\setlength{\headsep}{0mm}     % Set the distance between the header box and the top of the text

\usepackage{gvv-book}
\usepackage{gvv}
\usepackage{cite}
\usepackage{amsmath,amssymb,amsfonts,amsthm}
\usepackage{algorithmic}
\usepackage{graphicx}
\usepackage{textcomp}
\usepackage{xcolor}
\usepackage{txfonts}
\usepackage{listings}
\usepackage{enumitem}
\usepackage{mathtools}
\usepackage{gensymb}
\usepackage{comment}
\usepackage[breaklinks=true]{hyperref}
\usepackage{tkz-euclide} 
\usepackage{listings}
% \usepackage{gvv}                                        
\def\inputGnumericTable{}                                 
\usepackage[latin1]{inputenc}                                
\usepackage{color}                                            
\usepackage{array}                                            
\usepackage{longtable}                                       
\usepackage{calc}                                             
\usepackage{multirow}                                         
\usepackage{hhline}                                           
\usepackage{ifthen}                                           
\usepackage{lscape}
\begin{document}

\bibliographystyle{IEEEtran}
\vspace{3cm}

\title{8.1.11}
\author{EE24BTECH11063 - Y. Harsha Vardhan Reddy}
 \maketitle
% \newpage
% \bigskip
{\let\newpage\relax\maketitle}

\renewcommand{\thefigure}{\theenumi}
\renewcommand{\thetable}{\theenumi}
\setlength{\intextsep}{10pt} % Space between text and floats


\numberwithin{equation}{enumi}
\numberwithin{figure}{enumi}
\renewcommand{\thetable}{\theenumi}

\textbf{Question}:\\
Find the area of the region bounded by the curve $y^2=4x$ and the line $x=3$\\
\\ \\
\textbf{Solution:}\\
\begin{table}[H]
    \centering
    \begin{tabular}{|c|c|}
    \hline
    \textbf{Variable} & \textbf{Description}\\
    \hline
    $P_0$ & initial principal amount\\
    \hline
    $r$ & rate of increase per year\\
    \hline
    $t$ & time in years\\
    \hline 
    $C \& C_1$ & arbitrary constants\\
    \hline
    $P$ & principal at any time $t$\\
    \hline
\end{tabular}

    \caption{Variables used}
    \label{tab1-1.2-20}
\end{table}
\textbf{Theoritical Solution: }

The point of intersection of the line with the circle is $x_i=h+k_i m$,\\
where, $k_i$ is a constant and is calculated as follows:-
$$k_i=\frac{1}{m^\top Vm}\brak{-m^\top \brak{Vh+u}\pm \sqrt{\sbrak{m^\top \brak{Vh+u}} 2-g\brak{h}\brak{m^\top Vm}}}$$\\
Substituting the input parameters into $k_i$,\\
\begin{multline}
     k_i =\frac{1}{\myvec{0 
 & 1}\myvec{0 & 0 \\ 0 & 1}\myvec{0 \\ 1}}\brak{-\myvec{0&1}\brak{\myvec{0&0\\0&1}\myvec{3\\0}+\myvec{-2\\0}}}\pm \\
     \sqrt{\sbrak{\myvec{0&1}\brak{\myvec{0&0\\0&1}\myvec{3\\0}+\myvec{-2\\0}}}^2-g\brak{h}\brak{\myvec{0&1}\myvec{0&0\\0&1}\myvec{0\\1}}} 
\end{multline}
We get,\\
$k_i= \sqrt{12},-\sqrt{12}$\\
Substituting $k_i$ into $x_i=h+k_i m$ we get\\
\begin{align}
     x_1&=\myvec{3\\0}+\brak{\sqrt{12}}\myvec{0\\1}\\
    \implies x_1 &=\myvec{3\\\sqrt{12}}\\
    x_2 &=\myvec{2\\0}+\brak{-\sqrt{12}}\myvec{0\\1}\\
    \implies x_2&=\myvec{3\\-\sqrt{12}}
\end{align}Area of the region bounded by $y^2=4x$ and $x=3$,
\begin{align}
   & 2 \times \int_0^3\brak{\sqrt{4x}}\cdot dx\\
   &= 2 \times 2 \times\left[ \frac{x^{3/2}}{3/2} \right]_0^3 \\
    &= \frac{8}{3} \times \sqrt{27}\\
    &= 13.856
\end{align}
The area of the region bounded between the curve $y^2=4x$ and $x=3$ is 13.856 sq.units

\textbf{Computational Solution:}\\

Taking trapezoid-shaped strips of small area and adding them all up. Say we have to find the area of $y_{x}$ from $x=x_0$ to $x=x_n$, discretize the points on the $x$ axis $x_0, x_1, x_2, \dots, x_n$ such that they are equally spaced with the step size $h$. \\
Sum of all trapezoidal areas is given by,
\begin{align}
  A&=\frac{1}{2}h\brak{y\brak{x_1}+y\brak{x_0}}+ \frac{1}{2}h\brak{y\brak{x_2}+y\brak{x_1}}+\dots+\frac{1}{2}h\brak{y\brak{x_n}+y\brak{x_{n-1}}}\\
  &=h\sbrak{\frac{1}{2}\brak{y\brak{x_0}+y\brak{x_n}}+ y\brak{x_1}+\dots+y\brak{x_{n-1}}}
\end{align}
Let $A\brak{x_n}$ be the area enclosed by the curve $y\brak{x}$ from $x=x_0$ to $x=x_n$, $\brak{x_0, x_1, \dots x_n}$ be equidistant points with step-size $h$.
\begin{align}
  A\brak{x_n+h}=A\brak{x_n}+\frac{1}{2}h\brak{y\brak{x_n+h}+y\brak{x_n}}
\end{align}
We can repeat this till we get required area.\\
Discretizing the steps, making $A\brak{x_n}=A_n, y\brak{x_n}=y_n$ we get,
\begin{align}
 A_{n+1}=A_n+\frac{1}{2}h\brak{y_{n+1}+y_n}
\end{align}
We can write $y_{n+1}$ in terms of $y_n$ using first principle of derivative. $y_{n+1}=y_n+hy^{\prime}_n$
\begin{align}
  A_{n+1}&=A_n+\frac{1}{2}h\brak{\brak{y_{n}+hy^{\prime}_n}+y_n}\\
  A_{n+1}&=A_n+\frac{1}{2}h\brak{2y_n+hy^{\prime}_n}\\
  A_{n+1}&=A_n+hy_n+\frac{1}{2}h^2y^{\prime}_n\\
  x_{n+1}&=x_n+h
\end{align}
In the given question, $y_n=\sqrt{4x_n} $ and $y^{\prime}_n= -\frac{4}{\sqrt{x}}$\\
The general difference equation will be given by
\begin{align}
  A_{n+1}&=A_n+hy_n+\frac{1}{2}h^2y^{\prime}_n\\
  A_{n+1}&=A_n+h\brak{\sqrt{4x_n} }+\frac{1}{2}h^2\brak{-\frac{4}{\sqrt{x_n}}}\\
  x_{n+1}&=x_n+h
\end{align}
Iterating from $x_n=0$ to $x_n=3$ will return required area.(Upper half region) \\
The final result is multiplied by two include both half regions \\
Area obtained computationally: $13.856$ sq. units\\
Area obtained theoretically: $13.856$ sq. units
\begin{figure}[ht!]
   \centering
   \includegraphics[width=\columnwidth]{figs/Figure_1.png}
\end{figure}
\end{document}
