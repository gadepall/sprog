\let\negmedspace\undefined
\let\negthickspace\undefined
\documentclass[journal]{IEEEtran}
\usepackage[a5paper, margin=10mm, onecolumn]{geometry}
\usepackage{lmodern} % Ensure lmodern is loaded for pdflatex
\usepackage{tfrupee} % Include tfrupee package

\setlength{\headheight}{1cm} % Set the height of the header box
\setlength{\headsep}{0mm}     % Set the distance between the header box and the top of the text

\usepackage{gvv-book}
\usepackage{gvv}
\usepackage{cite}
\usepackage{amsmath,amssymb,amsfonts,amsthm}
\usepackage{algorithmic}
\usepackage{graphicx}
\usepackage{textcomp}
\usepackage{xcolor}
\usepackage{txfonts}
\usepackage{listings}
\usepackage{enumitem}
\usepackage{mathtools}
\usepackage{gensymb}
\usepackage{comment}
\usepackage[breaklinks=true]{hyperref}
\usepackage{tkz-euclide} 
\usepackage{listings}
\def\inputGnumericTable{}                                 
\usepackage[latin1]{inputenc}                                
\usepackage{color}                                            
\usepackage{array}                                            
\usepackage{longtable}                                       
\usepackage{calc}                                             
\usepackage{multirow}                                         
\usepackage{hhline}                                           
\usepackage{ifthen}                                           
\usepackage{lscape}

\begin{document}

\bibliographystyle{IEEEtran}
\vspace{3cm}

\title{12.8.1.12}
\author{EE24BTECH11030 - KEDARANANDA}
% \maketitle
% \newpage
% \bigskip
{\let\newpage\relax\maketitle}

\renewcommand{\thefigure}{\theenumi}
\renewcommand{\thetable}{\theenumi}
\setlength{\intextsep}{10pt} % Space between text and floats

\textbf{Question:}
 Area lying in the first quadrant and bounded by the circle $x^{2} + y^{2} = 4$ and the lines.x = 0 and x = 2 is\\

\textbf{Solution:}\\\\
\begin{table}[H]
    \centering
    \begin{tabular}{|c|c|}
    \hline
    \textbf{Variable} & \textbf{Description}\\
    \hline
    $P_0$ & initial principal amount\\
    \hline
    $r$ & rate of increase per year\\
    \hline
    $t$ & time in years\\
    \hline 
    $C \& C_1$ & arbitrary constants\\
    \hline
    $P$ & principal at any time $t$\\
    \hline
\end{tabular}

    \caption{Variables used}
\end{table} 
The point of intersection of the line with the circle is 
\begin{align}
    x_i = h + k_i m
\end{align}
where, $k_i$ is a constant and is calculated as follows:\\
\begin{align}
k_i = \frac{1}{m^\top Vm} \brak{-m^\top \brak{Vh + u} \pm \sqrt{\sbrak{m^\top \brak{Vh + u}}^2 - g\brak{h}\brak{m^\top Vm}}}.
\end{align}
Substituting the input parameters into $k_i$,\\

\begin{multline}
k_i = \frac{1}{\myvec{1 & 0} \myvec{1 & 0 \\ 0 & 1} \myvec{1 \\ 0}}
\brak{-\myvec{1 & 0} \brak{\myvec{1 & 0 \\ 0 & 1} \myvec{2 \\ 0} + \myvec{0 \\ 0}}} \pm \\
\sqrt{\sbrak{\myvec{1 & 0} \brak{\myvec{1 & 0 \\ 0 & 1} \myvec{2 \\ 0} + \myvec{0 \\ 0}}}^2 - g\brak{h}\brak{\myvec{1 & 0} \myvec{1 & 0 \\ 0 & 1} \myvec{1 \\ 0}}}.
\end{multline}

We get,\\
$k_i = 0, -4$.

Substituting $k_i$ into $x_i = h + k_i m$, we get\\

\begin{align}
x_1 &= \myvec{2 \\ 0} + \brak{0} \myvec{1 \\ 0} \\
&\implies x_1 = \myvec{2 \\ 0}, \\
x_2 &= \myvec{2 \\ 0} + \brak{-4} \myvec{1 \\ 0} \\
&\implies x_2 = \myvec{2 \\ 0} + \myvec{-4 \\ 0} \\
&\implies x_2 = \myvec{-2 \\ 0}.
\end{align}

The area of the region bounded by the circle $x^2 + y^2 = 4$, the line $x = 0$, and $x = 2$ in the first quadrant is:
$\text{Area} = \int_0^2 \sqrt{4 - x^2} \, dx$.

Using trigonometric substitution, we calculate:
\begin{align}
\text{Area} &= \int_0^2 \sqrt{4 - x^2} \, dx \\
&= \left[\frac{x}{2} \sqrt{4 - x^2} + 2 \sin^{-1}\brak{\frac{x}{2}}\right]_0^2 \\
&= \brak{\frac{2}{2} \sqrt{4 - 2^2} + 2 \sin^{-1}\brak{\frac{2}{2}}} - \brak{\frac{0}{2} \sqrt{4 - 0^2} + 2 \sin^{-1}\brak{\frac{0}{2}}} \\
&= \brak{0 + 2 \cdot \frac{\pi}{2}} - \brak{0 + 0} \\
&= \pi.
\end{align}

Thus, the area of the region is:
$\pi$.\\
\textbf{Computational Solution:}\\
Taking trapezoid-shaped strips of small area and adding them all up. Say we have to find the area of $y_{x}$ from $x=x_0$ to $x=x_n$, discretize the points on the $x$ axis $x_0, x_1, x_2, \dots, x_n$ such that they are equally spaced with the step size $h$. \\
Sum of all trapezoidal areas is given by,
\begin{align}
  A&=\frac{1}{2}h\brak{y\brak{x_1}+y\brak{x_0}}+ \frac{1}{2}h\brak{y\brak{x_2}+y\brak{x_1}}+\dots+\frac{1}{2}h\brak{y\brak{x_n}+y\brak{x_{n-1}}}\\
  &=h\sbrak{\frac{1}{2}\brak{y\brak{x_0}+y\brak{x_n}}+ y\brak{x_1}+\dots+y\brak{x_{n-1}}}
\end{align}
Let $A\brak{x_n}$ be the area enclosed by the curve $y\brak{x}$ from $x=x_0$ to $x=x_n$, $\brak{x_0, x_1, \dots x_n}$ be equidistant points with step-size $h$.
\begin{align}
  A\brak{x_n+h}=A\brak{x_n}+\frac{1}{2}h\brak{y\brak{x_n+h}+y\brak{x_n}}
\end{align}
We can repeat this till we get required area.\\
Discretizing the steps, making $A\brak{x_n}=A_n, y\brak{x_n}=y_n$ we get,
\begin{align}
 A_{n+1}=A_n+\frac{1}{2}h\brak{y_{n+1}+y_n}
\end{align}
We can write $y_{n+1}$ in terms of $y_n$ using first principle of derivative. $y_{n+1}=y_n+hy^{\prime}_n$
\begin{align}
  A_{n+1}&=A_n+\frac{1}{2}h\brak{\brak{y_{n}+hy^{\prime}_n}+y_n}\\
  A_{n+1}&=A_n+\frac{1}{2}h\brak{2y_n+hy^{\prime}_n}\\
  A_{n+1}&=A_n+hy_n+\frac{1}{2}h^2y^{\prime}_n\\
  x_{n+1}&=x_n+h
\end{align}
In the given question, $y_n=\sqrt{4-x_n^2}$ and $y^{\prime}_n= \frac{-x_n}{\brak{\sqrt{4-x_n^2}}}$\\
The general difference equation will be given by
\begin{align}
  A_{n+1}&=A_n+hy_n+\frac{1}{2}h^2y^{\prime}_n\\
  A_{n+1}&=A_n+h\brak{\sqrt{4-x_n^2}}+\frac{1}{2}h^2\brak{\frac{-x_n}{\brak{\sqrt{4-x_n^2}}}}\\
  x_{n+1}&=x_n+h
\end{align}
Iterating till we reach $x_n=2$ will return required area. \\
Area obtained computationally: $3.1416$ sq. units\\
Area obtained theoretically: $\pi$ sq. units = 3.1416 sq.unis
\begin{figure}[h!]
   \centering
   \includegraphics[width=1\columnwidth]{figs/Fig.png}
\end{figure}
\end{document}

