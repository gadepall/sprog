\documentclass[journal]{IEEEtran}
\usepackage[a5paper, margin=10mm, onecolumn]{geometry}
\usepackage{tfrupee} % Include tfrupee package

\setlength{\headheight}{1cm} % Set the height of the header box
\setlength{\headsep}{0mm}     % Set the distance between the header box and the top of the text

\usepackage{gvv-book}
\usepackage{gvv}
\usepackage{cite}
\usepackage{amsmath,amssymb,amsfonts,amsthm}
\usepackage{algorithmic}
\usepackage{graphicx}
\usepackage{textcomp}
\usepackage{xcolor}
\usepackage{txfonts}
\usepackage{listings}
\usepackage{enumitem}
\usepackage{mathtools}
\usepackage{gensymb}
\usepackage{comment}
\usepackage[breaklinks=true]{hyperref}
\usepackage{tkz-euclide} 
\usepackage{listings}
\def\inputGnumericTable{}                                 
\usepackage[latin1]{inputenc}                                
\usepackage{color}                                            
\usepackage{array}                                            
\usepackage{longtable}                                       
\usepackage{calc}                                             
\usepackage{multirow}                                         
\usepackage{hhline}                                           
\usepackage{ifthen}                                           
\usepackage{lscape}

\begin{document}

\bibliographystyle{IEEEtran}
\vspace{3cm}

\title{11.16.3.8.2}
\author{EE24BTECH11035 -K.Akhil}
\maketitle

\renewcommand{\thefigure}{\theenumi}
\renewcommand{\thetable}{\theenumi}
\setlength{\intextsep}{10pt} % Space between text and floats

\numberwithin{equation}{enumi}
\numberwithin{figure}{enumi}
\renewcommand{\thetable}{\theenumi}

\textbf{Question}:\\
If three coins are tossed once, what is the probability of getting exactly 2 heads?\\
\textbf{Solution: }\\
Define a discrete random variable $X = $ number of heads\\
We will take our random variable as a sum of outcomes of three Bernoulli random variables
\begin{align}
    X = X_1+X_2+X_3
\end{align}
Where
\begin{align}
X_i = 
\begin{cases}
    1, & \text{Outcome in Heads}\\
    0, & \text{Outcome in Tails}
\end{cases}\\
p_{X_i}(n) = 
\begin{cases}
    1-p, & n = 0\\
    p, & n = 1
\end{cases}
\end{align}
Where $p=\frac{1}{2}$\\
Using properties of Z-Transform of PMF
\begin{align}
    M_X(z) &= M_{X_1}(z)M_{X_2}(z)M_{X_3}(z)\\
    M_{X_1}(z) &= \sum_{n=-\infty}^{\infty}p_{X_1}(n)z^{-n} = (1-p)+pz^{-1}\\
    M_{X_2}(z) &= \sum_{n=-\infty}^{\infty}p_{X_2}(n)z^{-n} = (1-p)+pz^{-1}\\
    M_{X_3}(z) &= \sum_{n=-\infty}^{\infty}p_{X_3}(n)z^{-n} = (1-p)+pz^{-1}\\
    M_X(z) &= ((1-p)+pz^{-1})^3\\
     &= \sum_{n=-\infty}^{\infty}\comb{3}{n}(1-p)^{3-n}p^nz^{-n}\\
    p_{X}(n) &= \comb{3}{n}p^{n}(1-p)^{3-n}\\
    p_{X}(n) &= \frac{\comb{3}{n}}{8}
\end{align}
The Probability Mass Function (PMF) for the given random variable is
\begin{align}
p_X(n) =
\begin{cases}
    \frac{1}{8}, & n = 0 \\
    \frac{3}{8}, & n = 1 \\
    \frac{3}{8}, & n = 2 \\
    \frac{1}{8}, & n = 3 \\
\end{cases}
\end{align}
The probability of getting exactly 2 heads is
\begin{align}
  p_X(2) &= \frac{3}{8}
\end{align}

The Cumulative Distribution Function (CDF) for the given random variable is
\begin{align}
  F_{X}(n) = \sum_{k=-\infty}^n p_X(k) = \begin{cases}
    0, & n < 0\\
    \frac{1}{8}, & 0 \le n < 1\\
    \frac{4}{8}, & 1 \le n < 2\\
    \frac{7}{8}, & 2 \le n < 3\\
    1, & n \ge 3
  \end{cases}
\end{align}
To find the probability of getting exactly 2 heads using the CDF:
\begin{align}
  P(X = 2) &= F_X(2) - F_X(1) \\
  &= \frac{7}{8} - \frac{4}{8} \\
  &= \frac{3}{8}
\end{align}

The probability of getting exactly 2 heads is \(\frac{3}{8}\).

\textbf{Simulation Process:}\\
To run a simulation we need to generate random numbers with uniform probability, which is done as shown below:
\begin{enumerate}
  \item Generate a random number by calling \texttt{rand()}. It generates a random number between 0 and \texttt{RAND\_MAX}.
  \item Divide the generated number by \texttt{RAND\_MAX} so that it becomes a real number in the range [0, 1).
  \item If the number is less than $p$, take it as the event happened (heads), otherwise, the event did not happen (tails).
\end{enumerate}

\begin{figure}[h!]
   \centering
   \includegraphics[width=1\columnwidth]{figs/pmf.png}
    \caption{Probability Mass Function of given Random variable}
\end{figure}
\begin{figure}[h!]
   \centering
   \includegraphics[width=1\columnwidth]{figs/cdf.png}
    \caption{Cumulative Distribution Function of given Random variable}
\end{figure}
\end{document}

