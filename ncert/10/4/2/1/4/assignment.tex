\let\negmedspace\undefined
\let\negthickspace\undefined
\documentclass[journal]{IEEEtran}
\usepackage[a5paper, margin=10mm, onecolumn]{geometry}
\usepackage{tfrupee}

\setlength{\headheight}{1cm}
\setlength{\headsep}{0mm}

\usepackage{gvv-book}
\usepackage{gvv}
\usepackage{cite}
\usepackage{amsmath,amssymb,amsfonts,amsthm}
\usepackage{algorithmic}
\usepackage{graphicx}
\usepackage{textcomp}
\usepackage{xcolor}
\usepackage{txfonts}
\usepackage{listings}
\usepackage{enumitem}
\usepackage{mathtools}
\usepackage{gensymb}
\usepackage{comment}
\usepackage[breaklinks=true]{hyperref}
\usepackage{tkz-euclide} 
\usepackage{listings}

\def\inputGnumericTable{}
\usepackage[latin1]{inputenc}                                
\usepackage{color}                                            
\usepackage{array}                                            
\usepackage{longtable}                                       
\usepackage{calc}                                             
\usepackage{multirow}                                         
\usepackage{hhline}                                           
\usepackage{ifthen}                                           
\usepackage{lscape}

\begin{document}

\bibliographystyle{IEEEtran}
\vspace{3cm}

\title{CHAPTER - 4\\Quadratic Equations}
\author{EE24BTECH11061 - Rohith Sai}

{\let\newpage\relax\maketitle}

\renewcommand{\thefigure}{\theenumi}
\renewcommand{\thetable}{\theenumi}
\setlength{\intextsep}{10pt}

\numberwithin{figure}{enumi}
\renewcommand{\thetable}{\theenumi}

\section*{Exercise : 4.2}
\begin{enumerate}
\item [4.1)] Find the roots of the following equation $2x^2 - x + \frac{1}{8} = 0$\\
\textbf{Solution:} \\
First, we simplify the given equation:
\begin{align}
    2x^2 - x + \frac{1}{8} = 0 \\
    \implies 16x^2 - 8x + 1 = 0
\end{align}

\section*{Companion Matrix}
For a quadratic equation of the form:
\begin{align}
    ax^2 + bx + c = 0,
\end{align}
the corresponding companion matrix is given by:
\begin{align}
    A = 
    \myvec{
        0 & 1 \\
        -\frac{c}{a} & -\frac{b}{a}
    }.
\end{align}

Substitute the coefficients $a = 16$, $b = -8$, and $c = 1$ into the companion matrix formula:
\begin{align}
    A = 
    \myvec{
        0 & 1 \\
        -\frac{1}{16} & \frac{8}{16}
    }
    =
    \myvec{
        0 & 1 \\
        -\frac{1}{16} & \frac{1}{2}
    }.
\end{align}

\section*{QR Algorithm}
The QR algorithm iteratively decomposes the matrix $A_n$ into an orthogonal matrix $Q_n$ and an upper triangular matrix $R_n$, and updates the matrix as:
\begin{align}
    A_{n+1} = R_n Q_n.
\end{align}
This process continues until $A_n$ converges to an upper triangular matrix, where the diagonal elements are the eigenvalues of $A$.

\subsection*{Steps of the Algorithm}
\begin{enumerate}
    \item Initialize the companion matrix:
    \begin{align}
        A_0 = 
        \myvec{
            0 & 1 \\
            -\frac{1}{16} & \frac{1}{2}
        }.
    \end{align}
    \item Perform QR decomposition of $A_n$:
    \begin{align}
        A_n = Q_n R_n,
    \end{align}
    where $Q_n$ is orthogonal and $R_n$ is upper triangular.
    \item Update the matrix:
    \begin{align}
        A_{n+1} = R_n Q_n.
    \end{align}
    \item Repeat the above steps until $A_n$ converges to an upper triangular matrix. The diagonal elements of this matrix are the eigenvalues, which correspond to the roots of the quadratic equation.
\end{enumerate}

\subsection*{Roots of the Quadratic Equation}
The eigenvalues of the companion matrix $A$ are the roots of the quadratic equation. Applying the QR algorithm numerically to $A$, we find:
\begin{align}
    \lambda_1 = 0.25, \quad \lambda_2 = 0.25.
\end{align}

\subsection*{Conclusion}
The QR decomposition method applied to the companion matrix of $2x^2 - x + \frac{1}{8} = 0$ finds the roots of the equation. Both roots are real and equal:
\begin{align}
    x = 0.25.
\end{align}
This demonstrates the utility of the QR algorithm in computing eigenvalues, which are the roots of polynomial equations.


\section*{Newton's Method}
Newton's Method is given by the update formula:
\begin{align}
    x_{n+1} = x_n - \frac{f(x_n)}{f'(x_n)}
\end{align}
where:
\begin{align}
    f(x) = 16x^2 - 8x + 1 \quad \text{and} \quad f'(x) = 32x - 8
\end{align}
The update equation becomes:
\begin{align}
    x_{n+1} = x_n - \frac{16x_n^2 - 8x_n + 1}{32x_n - 8}
\end{align}
Using an initial guess $x_0 = 0.5$, we observe that $x_n$ converges at the 18th iteration to:
\begin{align}
    x = 0.25
\end{align}

\section*{Secant Method}
Alternatively, we can use the Secant Method, which avoids the derivative:
\begin{align}
    x_{n+1} = x_n + f(x_n) \frac{x_n - x_{n-1}}{f(x_n) - f(x_{n-1})}
\end{align}
Taking initial guesses $x_0 = 0.5$ and $x_1 = 0.4$, we observe that $x_n$ converges at the 25th iteration to:
\begin{align}
    x = 0.25
\end{align}
The graph below shows the equation and the root of the equation
\begin{figure}[H]
    \centering
    \includegraphics[width=\columnwidth]{figs/fig.png}
\end{figure}

\end{enumerate}
\end{document}
