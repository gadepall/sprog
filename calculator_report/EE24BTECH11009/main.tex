\documentclass{article}
\usepackage[utf8]{inputenc}
\usepackage{graphicx}
\usepackage{tabularx}
\usepackage{amsmath}
\usepackage{hyperref}
\usepackage{xcolor}
\usepackage{listings}
\usepackage{booktabs}
\usepackage{geometry}
\usepackage{float}

\geometry{a4paper, margin=1in}

% Custom colors
\definecolor{codegreen}{rgb}{0,0.6,0}
\definecolor{codegray}{rgb}{0.5,0.5,0.5}
\definecolor{codepurple}{rgb}{0.58,0,0.82}
\definecolor{backcolour}{rgb}{0.95,0.95,0.92}

% Listing style
\lstdefinestyle{mystyle}{
    backgroundcolor=\color{backcolour},
    commentstyle=\color{codegreen},
    keywordstyle=\color{magenta},
    numberstyle=\tiny\color{codegray},
    stringstyle=\color{codepurple},
    basicstyle=\ttfamily\footnotesize,
    breakatwhitespace=false,
    breaklines=true,
    captionpos=b,
    keepspaces=true,
    numbers=left,
    numbersep=5pt,
    showspaces=false,
    showstringspaces=false,
    showtabs=false,
    tabsize=4,
    frame=single,
    language=C
}

\lstset{style=mystyle}

\title{AVR Scientific Calculator Project Report}
\author{Mokshith Kumar Reddy}

\begin{document}
{\let\newpage\relax\maketitle}

\renewcommand{\thefigure}{\theenumi}
\renewcommand{\thetable}{\theenumi}
\setlength{\intextsep}{10pt} % Space between text and floats


\numberwithin{equation}{enumi}
\numberwithin{figure}{enumi}
\renewcommand{\thetable}{\theenumi}
\tableofcontents
\newpage

\section{Introductions}
This report consists of the data how to make a scientific calculator(can handle algebraic, trigonometric, logarithmic functions) from the components shown below.
\section{Components Required}
\begin{itemize}
\item Arduino Uno/Nano
\item LCD-Display
\item Push Buttons
\item Breadboard and Wires
\end{itemize}

\section{Hardware Design}

\subsection{Component Connections}

\begin{table}[H]
\centering
\caption{Hardware Connections}
\label{tab:connections}
\begin{tabular}{llll}
\toprule
\textbf{Component} & \textbf{MCU Pin} & \textbf{Arduino Pin} & \textbf{Function} \\
\midrule
LCD RS & PD0 & D0 & Register Select \\
LCD E & PD1 & D1 & Enable \\
LCD D4-D7 & PD2-PD5 & D2-D5 & Data Bus \\
Keypad ROW1 & PD6 & D6 & Row 1 \\
Keypad ROW2 & PD7 & D7 & Row 2 \\
Keypad ROW3 & PB0 & D8 & Row 3 \\
Keypad ROW4 & PB1 & D9 & Row 4 \\
Keypad COL1 & PB2 & D10 & Column 1 \\
Keypad COL2 & PB3 & D11 & Column 2 \\
Keypad COL3 & PB4 & D12 & Column 3 \\
\bottomrule
\end{tabular}
\end{table}
\subsection{Matrix Scanning}
Here we are implementing the matrix method keyboard scanning the 12 different push buttons using only 7 ports as follows, we arrange the buttons in the matrix form of $4\times3$ and we connect every lower wire of buttons in a row and every upper wire of buttons in the column is connected so if we enable a button a unique set of signal is given to Arduino. So now we need conserve the ports.
\section{Software Implementation}

\subsection{Main Program Structure}

\begin{lstlisting}[caption=Main Program Loop, label=lst:main]
#define F_CPU 16000000UL
#include <avr/io.h>
#include <util/delay.h>

int main(void) {
    // Initialize hardware
    DDRD = 0xFF;  // Set PORTD as outputs
    DDRB = 0x03;  // Set PB0-PB1 as outputs
    
    // Initialize LCD
    LCD_Init();
    LCD_Message("Calculator Ready");
    
    while(1) {
        // Check mode buttons
        if (!(PINC & (1<<PC2))) toggleTrigMode();
        if (!(PINC & (1<<PC3))) calculateResult();
        
        // Handle keypad input
        char key = getKeyPressed();
        if (key != '\0') {
            handleKeyPress(key);
            _delay_ms(300);  // Debounce delay
        }
    }
    return 0;
}
\end{lstlisting}

\subsection{Keypad Scanning}

\begin{lstlisting}[caption=Keypad Scanning Function, label=lst:keypad]
const char keys[4][3] = {
    {'1','2','3'},
    {'4','5','6'},
    {'7','8','9'},
    {'A','0','C'}
};

char getKeyPressed() {
    for (uint8_t row = 0; row < 4; row++) {
        // Activate current row
        switch(row) {
            case 0: PORTD &= ~(1<<ROW1); break;
            case 1: PORTD &= ~(1<<ROW2); break;
            case 2: PORTB &= ~(1<<ROW3); break;
            case 3: PORTB &= ~(1<<ROW4); break;
        }
        _delay_us(10);
        
        // Check columns
        if (!(PINB & (1<<COL1))) return keys[row][0];
        if (!(PINB & (1<<COL2))) return keys[row][1];
        if (!(PINB & (1<<COL3))) return keys[row][2];
        
        // Deactivate row
        switch(row) {
            case 0: PORTD |= (1<<ROW1); break;
            case 1: PORTD |= (1<<ROW2); break;
            case 2: PORTB |= (1<<ROW3); break;
            case 3: PORTB |= (1<<ROW4); break;
        }
    }
    return '\0';  // No key pressed
}
\end{lstlisting}

\subsection{LCD Interface}

\begin{lstlisting}[caption=LCD Initialization, label=lst:lcd]
void LCD_Init() {
    _delay_ms(50);
    SendNibble(0x03);
    _delay_ms(5);
    SendNibble(0x03);
    _delay_us(100);
    SendNibble(0x02);  // 4-bit mode
    
    LCD_Cmd(0x28);  // 2 lines, 5x8 matrix
    LCD_Cmd(0x0C);  // Display on, cursor off
    LCD_Cmd(0x06);  // Increment cursor
    LCD_Cmd(0x01);  // Clear display
    _delay_ms(2);
}

void LCD_Char(uint8_t data) {
    PORTD |= (1<<LCD_RS);  // Set to data mode
    SendByte(data);
}

void LCD_Message(const char *text) {
    while(*text) LCD_Char(*text++);
}
\end{lstlisting}

\subsection{Mathematical Operations}

\begin{lstlisting}[caption=Trigonometric Functions, label=lst:trig]
float sin_euler(float x) {
    x = x * PI / 180;  // Convert to radians
    float term = x, sum = x;
    
    for(int n = 3; n < 15; n += 2) {
        term *= -x*x/(n*(n-1));
        sum += term;
    }
    return sum;
}

float cos_euler(float x) {
    x = x * PI / 180;  // Convert to radians
    float term = 1, sum = 1;
    
    for(int n = 2; n < 15; n += 2) {
        term *= -x*x/(n*(n-1));
        sum += term;
    }
    return sum;
}
\end{lstlisting}

\section{Conclusion}
The AVR scientific calculator project successfully demonstrates:
Efficient keypad scanning using matrix techniques
, Clear output on LCD display, Accurate mathematical computations
and  Responsive user interface

Future enhancements could include:
\begin{itemize}
\item Floating-point optimization
\item Additional scientific functions
\item Graphical display capabilities
\end{itemize}

\end{document}