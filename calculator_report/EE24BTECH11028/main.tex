
\documentclass[12pt,a4paper]{article}
\usepackage[a5paper, margin=10mm, onecolumn]{geometry}
\usepackage{tfrupee} % Include tfrupee package
\usepackage{gvv-book}
\usepackage{gvv}
\usepackage{cite}
\usepackage{amsmath,amssymb,amsfonts,amsthm}
\usepackage{algorithmic}
\usepackage{graphicx}
\usepackage{textcomp}
\usepackage{xcolor}
\usepackage{txfonts}
\usepackage{booktabs}  % For professional-looking tables
\usepackage{array}
\usepackage{listings}
\usepackage{enumitem}
\usepackage{mathtools}
\usepackage{gensymb}
\usepackage{comment}
\usepackage{tkz-euclide}
\usepackage{circuitikz}
\usepackage{array}
\usepackage{longtable}
\usepackage{multirow}
\usepackage{hhline}
\usepackage{ifthen}
\usepackage{lscape}
\usepackage{hyperref}
\usepackage{graphicx} % For including images
\usepackage{geometry} % For adjusting page margins
\usepackage{amsmath}  % For mathematical equations
\usepackage{listings} % For code listings
\usepackage{xcolor}   % For syntax highlighting in code
\usepackage{hyperref} % For hyperlinks

\geometry{a4paper, margin=1in}
\hypersetup{breaklinks=true}

% Arduino code styling
\lstset{
    language=C++,
    basicstyle=\ttfamily\small,
    keywordstyle=\color{blue},
    commentstyle=\color{green},
    stringstyle=\color{red},
    numbers=left,
    numberstyle=\tiny\color{gray},
    stepnumber=1,
    numbersep=5pt,
    backgroundcolor=\color{white},
    showspaces=false,
    showstringspaces=false,
    showtabs=false,
    tabsize=2,
    captionpos=b,
    breaklines=true,
    breakatwhitespace=true,
    frame=single
}

\title{SCIENTIFIC CALCULATOR}
\author{JADHAV RAJESH - EE24BTECH11028}

\begin{document}

\maketitle

\section*{Objective}
To design and implement a scientific calculator using an Arduino Uno board programmed with AVR-GCC, capable of performing basic and advanced mathematical operations.

\section*{Materials Required}
\subsection{Hardware Components}
\begin{itemize}
    \item Arduino Uno (ATmega328P microcontroller)
    \item 16×2 LCD Display (for output)
    \item 4×4 Matrix Keypad (for input)
    \item Breadboard \& Jumper Wires
    \item Potentiometer (for LCD contrast adjustment)
    \item USB Cable (for programming)
\end{itemize}

\subsection{Software Tools}
\begin{itemize}
    \item Arduino IDE / AVR-GCC Toolchain
    \item AVRDUDE (for flashing the program)
\end{itemize}
\section*{Circuit design}
\begin{table}[H]
\centering
\caption{Push Button Designations}
\label{tab:functions}
\begin{tabular}{ccc}
\toprule
Button Number & Function \\
\midrule
1 - 10 & Digits 0 - 9 \\
11 & Clear \\
12 & $\ln{(x)}$ and $\log{(x)}$ \\
13 & Right Parenthesis \\
14 & $\sin{(x)}$, $\cos{(x)}$, and $\tan{(x)}$ \\
15 & $e$ and $\pi$ \\
16 & Backspace \\
17 & Decimal Point \\
18 & Equal To \\
19 & Left Parenthesis \\
20 & Division \\
21 & Multiplications \\
22 & Subtraction \\
23 & Addition \\
\bottomrule
\end{tabular}
\end{table}
% \newpage
\section*{Experimental Procedure}
\subsection{Circuit Setup}
\subsubsection{LCD Interfacing with Arduino}
Connect the \textbf{16×2 LCD} to Arduino Uno as follows:
\begin{itemize}
    \item \textbf{VSS (GND)} → Arduino GND
    \item \textbf{VDD (5V)} → Arduino 5V
    \item \textbf{VO (Contrast)} → Potentiometer middle pin
    \item \textbf{RS (Register Select)} → Arduino Pin 12
    \item \textbf{RW (Read/Write)} → GND (for write mode)
    \item \textbf{EN (Enable)} → Arduino Pin 11
    \item \textbf{D4-D7 (Data Pins)} → Arduino Pins 5, 4, 3, 2
    \item \textbf{Backlight (A \& K)} → 5V \& GND (with resistor if needed)
\end{itemize}
\newpage
\begin{table}[H]
    \centering
    \renewcommand{\arraystretch}{1.2} % Adjust row height
    \begin{tabular}{|c|c|}
        \hline
        \textbf{Component} & \textbf{Arduino Pin} \\
        \hline
        \multicolumn{2}{|c|}{\textbf{Button Matrix}} \\
        \hline
        Row 1 & 2 \\
        Row 2 & 3 \\
        Row 3 & 4 \\
        Row 4 & 5 \\
        Column 1 & 6 \\
        Column 2 & 7 \\
        Column 3 & 8 \\
        Column 4 & 9 \\
        Column 5 & 10 \\
        \hline
        \multicolumn{2}{|c|}{\textbf{Shift Button}} \\
        \hline
        Shift Button & 13 \\
        GND & GND \\
        \hline
        \multicolumn{2}{|c|}{\textbf{LCD Display (16x2, Non-I2C)}} \\
        \hline
        LCD RS & A0 \\
        LCD EN & A1 \\
        LCD D4 & A2 \\
        LCD D5 & A3 \\
        LCD D6 & A4 \\
        LCD D7 & A5 \\
        \hline
    \end{tabular}
    \caption{Circuit Connections of the Scientific Calculator}
    \label{tab:circuit_connections}
\end{table}
\newpage

\subsubsection{Keypad Interfacing}
Connect the \textbf{4×4 Matrix Keypad} to Arduino:
\begin{itemize}
    \item Rows → Arduino Pins 6, 7, 8, 9
    \item Columns → Arduino Pins A0, A1, A2, A3
\end{itemize}

\subsection{Software Development (AVR-GCC Programming)}
\subsubsection{Step 1: Initialize LCD and Keypad}
\begin{itemize}
    \item Use the \texttt{avr/io.h} library for GPIO control.
    \item Implement \textbf{LiquidCrystal} library functions (or custom functions) for LCD.
    \item Write a keypad scanning function to detect button presses.
\end{itemize}

\subsubsection{Step 2: Implement Calculator Logic}
\begin{itemize}
    \item \textbf{Basic Operations (+, -, *, /)}  
        - Store operands and operator.  
        - Compute result using arithmetic logic.  
    \item \textbf{Scientific Functions (sin, cos, tan, log, sqrt, etc.)}  
        - Use \texttt{<math.h>} library for advanced computations.  
    \item \textbf{Error Handling}  
        - Check for division by zero, invalid inputs.  
\end{itemize}

\subsubsection*{Step 3: Display Results}
\begin{itemize}
    \item Print input and output on the LCD.
    \item Clear display when "C" (clear) is pressed.
\end{itemize}
\subsection{Testing \& Validation}
\begin{itemize}
    \item Test each operation (basic \& scientific).
    \item Verify correct display on LCD.
    \item Debug if any issues arise (e.g., wrong calculations, unresponsive keypad).
\end{itemize}

\section*{Expected Results}
The Arduino-based scientific calculator should:
\begin{itemize}
    \item Accept inputs via the keypad.
    \item Perform arithmetic and scientific operations accurately.
    \item Display results on the LCD.
\end{itemize}

\section*{Conclusion}
This experiment demonstrates the development of a \textbf{scientific calculator using Arduino Uno with AVR-GCC}, integrating hardware (LCD, keypad) and software (mathematical logic). The system should function reliably for engineering and educational applications.


\end{document}
