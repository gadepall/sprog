\documentclass[12pt,a4paper]{article}
\usepackage{amsmath}
\usepackage{amssymb}
\usepackage{listings}
\usepackage{xcolor}
\usepackage{graphicx}
\usepackage{hyperref}

% Define colors for code highlighting
\definecolor{codegreen}{rgb}{0,0.6,0}
\definecolor{codegray}{rgb}{0.5,0.5,0.5}
\definecolor{codepurple}{rgb}{0.58,0,0.82}
\definecolor{backcolour}{rgb}{0.95,0.95,0.92}

% Code listing style
\lstdefinestyle{mystyle}{
    backgroundcolor=\color{backcolour},   
    commentstyle=\color{codegreen},
    keywordstyle=\color{magenta},
    numberstyle=\tiny\color{codegray},
    stringstyle=\color{codepurple},
    basicstyle=\ttfamily\footnotesize,
    breakatwhitespace=false,         
    breaklines=true,                 
    captionpos=b,                    
    keepspaces=true,                 
    numbers=left,                    
    numbersep=5pt,                  
    showspaces=false,                
    showstringspaces=false,
    showtabs=false,                  
    tabsize=2
}

\lstset{style=mystyle}

\title{Scientific Calculator Implementation on Arduino Microcontroller}
\author{Balaji B}
\date{\today}

\begin{document}

\maketitle

\begin{abstract}
This report details the design and development of a scientific calculator powered by an Arduino microcontroller. The calculator supports a wide range of mathematical operations, including trigonometric, logarithmic, and exponential functions, along with numerical methods for solving differential equations. To ensure high precision, the implementation utilizes the fourth-order Runge-Kutta method for computations.
\end{abstract}

\tableofcontents

\section{Introduction}
The goal of this scientific calculator project is to execute various mathematical functions using an Arduino microcontroller with constrained computational power. Rather than using built-in math libraries, numerical methods have been employed to calculate functions such as sine, cosine, logarithms, and exponentials.

\section{Hardware Components}
\begin{itemize}
    \item Arduino Microcontroller 
    \item 16x2 LCD Display
    \item 4x5 Matrix Keypad for input
    \item Additional electronic components for support
\end{itemize}

\section{Mathematical Methods}

\subsection{Runge-Kutta 4th Order Method}
The Runge-Kutta 4th order method (RK4) is used to solve ordinary differential equations (ODEs). It provides a good balance between accuracy and computational complexity. For a first-order ODE of the form:

\begin{equation}
\frac{dy}{dx} = f(x, y)
\end{equation}

The RK4 method approximates the solution using:

\begin{align}
y_{n+1} &= y_n + \frac{1}{6}(k_1 + 2k_2 + 2k_3 + k_4) \\
k_1 &= h \cdot f(x_n, y_n) \\
k_2 &= h \cdot f(x_n + \frac{h}{2}, y_n + \frac{k_1}{2}) \\
k_3 &= h \cdot f(x_n + \frac{h}{2}, y_n + \frac{k_2}{2}) \\
k_4 &= h \cdot f(x_n + h, y_n + k_3)
\end{align}

where $h$ is the step size.

\subsection{Computing Logarithmic Functions}
The natural logarithm is computed using the integral definition:

\begin{equation}
\ln(x) = \int_{1}^{x} \frac{1}{t} dt
\end{equation}

This is solved using the RK4 method with the differential equation:

\begin{equation}
\frac{dy}{dx} = \frac{1}{x}
\end{equation}

\subsection{Computing Exponential Functions}
The exponential function $e^x$ is computed by solving the ODE:

\begin{equation}
\frac{dy}{dx} = y
\end{equation}

With the initial condition $y(0) = 1$.

\subsection{Computing Trigonometric Functions}
Sine and cosine functions are computed by solving the second-order ODE:

\begin{equation}
\frac{d^2y}{dx^2} = -y
\end{equation}

This is converted to a system of first-order ODEs:

\begin{align}
\frac{dy_1}{dx} &= y_2 \\
\frac{dy_2}{dx} &= -y_1
\end{align}

With initial conditions:
\begin{itemize}
    \item For sine: $y_1(0) = 0, y_2(0) = 1$
    \item For cosine: $y_1(0) = 1, y_2(0) = 0$
\end{itemize}


\subsection{Fast Inverse Square Root}
For computing inverse square root $\frac{1}{\sqrt{x}}$, we use an optimized algorithm that provides a good approximation:

\begin{equation}
y_{n+1} = y_n \cdot \left(\frac{3 - x \cdot y_n^2}{2}\right)
\end{equation}

\section{Implementation Details}

\subsection{Code Structure}
The code is structured into distinct functional modules, each responsible for a specific aspect of the implementation:
\begin{itemize}
    \item \textbf{Numerical Methods:} Implements algorithms for mathematical computations.
    \item \textbf{LCD Interface:} Manages communication with the LCD display.
    \item \textbf{Keypad Interface:} Handles user input through the keypad.
\end{itemize}

\subsection{Numerical Methods Implementation}
The Runge-Kutta fourth-order (RK4) method is implemented to solve both first-order and second-order differential equations efficiently.

\url{https://github.com/Balaji29-code/Hardware_Project/blob/main/scientific_calculator/codes/range_kutta1.c} \\

\url{https://github.com/Balaji29-code/Hardware_Project/blob/main/scientific_calculator/codes/range_kutta2.c}

\subsection{User Interface}
The calculator supports three operational modes, each offering different functionalities:  
\begin{itemize}
    \item \textbf{Normal Mode:} Performs basic arithmetic operations.  
    \item \textbf{Alpha Mode:} Provides access to scientific functions such as sine, cosine, and logarithm.  
    \item \textbf{Shift Mode:} Enables advanced functions, including inverse trigonometric calculations and memory operations.  
\end{itemize}

\section{Mathematical Functions}
The calculator implements the following mathematical functions:

\subsection{Power and Root Functions}
\begin{itemize}
    \item $\sqrt{x}$: Computed using RK4 for the ODE $\frac{dy}{dx} = \frac{1}{2y}$ with initial condition $y(1) = 1$
    \item $x^y$: Computed using iterative multiplication
    \item $\sqrt[3]{x}$: Computed using Newton's method
\end{itemize}


\subsection{Logarithmic and Exponential Functions}  
The calculator implements logarithmic and exponential functions using numerical methods:  
\begin{itemize}  
    \item \textbf{Natural Logarithm} ($\ln(x)$): Computed via numerical integration of $\frac{1}{x}$.  
    \item \textbf{Base-10 Logarithm} ($\log_{10}(x)$): Evaluated using the relation $\log_{10}(x) = \frac{\ln(x)}{\ln(10)}$.  
    \item \textbf{Exponential Function} ($e^x$): Solved using the Runge-Kutta fourth-order (RK4) method for the differential equation $\frac{dy}{dx} = y$ with the initial condition $y(0) = 1$.  
    \item \textbf{Power of 10} ($10^x$): Computed using the general power function.  
\end{itemize}  

\subsection{Trigonometric Functions}  
The calculator computes trigonometric and inverse trigonometric functions using numerical methods:  

\begin{itemize}  
    \item \textbf{Sine} ($\sin(x)$): Solved using the Runge-Kutta fourth-order (RK4) method for the differential equation $\frac{d^2y}{dx^2} = -y$ with initial conditions $y(0) = 0$, $y'(0) = 1$.  
    \item \textbf{Cosine} ($\cos(x)$): Computed using RK4 for the equation $\frac{d^2y}{dx^2} = -y$ with initial conditions $y(0) = 1$, $y'(0) = 0$.  
    \item \textbf{Tangent} ($\tan(x)$): Evaluated as $\frac{\sin(x)}{\cos(x)}$.  
    \item \textbf{Inverse Sine} ($\sin^{-1}(x)$): Determined via numerical integration of $\frac{1}{\sqrt{1-x^2}}$.  
    \item \textbf{Inverse Cosine} ($\cos^{-1}(x)$): Computed using the identity $\cos^{-1}(x) = \frac{\pi}{2} - \sin^{-1}(x)$.  
    \item \textbf{Inverse Tangent} ($\tan^{-1}(x)$): Calculated using numerical integration of $\frac{1}{1+x^2}$.  
\end{itemize}  

\section{Conclusion}  
The implementation of the scientific calculator showcases how complex mathematical functions can be efficiently computed on resource-limited microcontrollers using numerical methods. The use of the Runge-Kutta method allows for accurate approximations of differential equations, enabling the calculation of transcendental functions with high precision.  

\subsection{Future Improvements}  
Potential enhancements for the scientific calculator include:  

\begin{itemize}  
    \item Expanding functionality to support advanced operations such as hyperbolic and statistical functions.  
    \item Enhancing the expression parser to accommodate more complex mathematical expressions.  
    \item Optimizing code efficiency for improved performance and reduced memory usage.  
    \item Integrating graphing capabilities for visualizing functions and equations.  
\end{itemize}  

\section{Full Source Code}
The complete source code for the scientific calculator is available in the project repository.

\url{https://github.com/Balaji29-code/Hardware_Project/tree/main/scientific_calculator/codes}
\end{document}

