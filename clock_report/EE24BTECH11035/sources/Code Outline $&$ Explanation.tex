
\definecolor{codebg}{rgb}{0.95,0.95,0.95}
\lstset{
    backgroundcolor=\color{codebg},
    basicstyle=\ttfamily\footnotesize,
    keywordstyle=\color{blue}\bfseries,
    commentstyle=\color{green!50!black},
    numbers=left,
    numberstyle=\tiny\color{gray},
    frame=single,
    breaklines=true
}
\subsection*{Defining CPU Frequency}
\begin{lstlisting}
#define F_CPU 16000000UL
\end{lstlisting}
The microcontroller runs at 16 MHz. This definition ensures proper timing calculations.

\subsection*{Including Required Headers}
\begin{lstlisting}
#include <avr/io.h>
#include <avr/interrupt.h>
#include <util/delay.h>
\end{lstlisting}
These headers provide functions for I/O operations, interrupt handling, and delays.

\subsection*{Defining BCD and Display Control Pins}
\begin{lstlisting}
#define A PD2  
#define B PD3  
#define C PD4  
#define D PD5  

#define H1 PD6  
#define H2 PD7  
#define M1 PB0  
#define M2 PB1  
#define S1 PB2  
#define S2 PB3  
\end{lstlisting}
- **PD2–PD5** are connected to the 7447 BCD decoder.
- **PD6, PD7, PB0–PB3** control which digit is active.

\subsection*{Clock Variables in BCD Format}
\begin{lstlisting}
volatile uint8_t hours = 0b00010010;
volatile uint8_t minutes = 0b00000000;
volatile uint8_t seconds = 0b00000000;
\end{lstlisting}
The time is stored in **Binary-Coded Decimal (BCD)** format.

\subsection*{Displaying a Single Digit}
\begin{lstlisting}
void displayDigit(uint8_t digit) {
    PORTD = (PORTD & 0b11000011) | (digit << 0b00000010);
}
\end{lstlisting}
This function sends a **BCD digit** to **PORTD (PD2–PD5)** while preserving other bits.

\subsection*{Displaying the Complete Time}
\begin{lstlisting}
void displayTime() {
    uint8_t h1 = (hours >> 4) & 0x0F;
    uint8_t h2 = hours & 0x0F;
    uint8_t m1 = (minutes >> 4) & 0x0F;
    uint8_t m2 = minutes & 0x0F;
    uint8_t s1 = (seconds >> 4) & 0x0F;
    uint8_t s2 = seconds & 0x0F;

    PORTD |= (1 << H1); displayDigit(h1); _delay_ms(5); PORTD &= ~(1 << H1);
    PORTD |= (1 << H2); displayDigit(h2); _delay_ms(5); PORTD &= ~(1 << H2);
    PORTB |= (1 << M1); displayDigit(m1); _delay_ms(5); PORTB &= ~(1 << M1);
    PORTB |= (1 << M2); displayDigit(m2); _delay_ms(5); PORTB &= ~(1 << M2);
    PORTB |= (1 << S1); displayDigit(s1); _delay_ms(5); PORTB &= ~(1 << S1);
    PORTB |= (1 << S2); displayDigit(s2); _delay_ms(5); PORTB &= ~(1 << S2);
}
\end{lstlisting}
- Extracts each **tens** and **units** place digit using bitwise operations.
- Uses **multiplexing** to display each digit sequentially.

\subsection*{Timer Interrupt for 1-Second Updates}
\begin{lstlisting}
ISR(TIMER1_COMPA_vect) {
    seconds += 1;
    if ((seconds & 0x0F) > 9) { 
        seconds = (seconds & 0xF0) + 0x10;
    }
    if (seconds >= 0x60) {
        seconds = 0x00;
        minutes += 1;
    }
    if ((minutes & 0x0F) > 9) { 
        minutes = (minutes & 0xF0) + 0x10;
    }
    if (minutes >= 0x60) {
        minutes = 0x00;
        hours += 1;
    }
    if ((hours & 0x0F) > 9) { 
        hours = (hours & 0xF0) + 0x10;
    }
    if (hours >= 0x24) {
        hours = 0x00;
    }
}
\end{lstlisting}
- Increments **seconds** every timer interrupt (1 second).
- **Handles carry propagation** from seconds → minutes → hours.

\subsection*{Timer1 Configuration}
\begin{lstlisting}
void timer1_init() {
    TCCR1B |= (1 << WGM12) | (1 << CS12) | (1 << CS10);
    OCR1A = 15624;
    TIMSK1 |= (1 << OCIE1A);
    sei();
}
\end{lstlisting}
- **Sets Timer1 to CTC mode**.
- **Prescaler 1024** → Results in **1-second intervals**.
- Enables **interrupts** on compare match.

\subsection*{Main Function}
\begin{lstlisting}
int main(void) {
    DDRD |= (1 << A) | (1 << B) | (1 << C) | (1 << D);
    DDRD |= (1 << H1) | (1 << H2);
    DDRB |= (1 << M1) | (1 << M2) | (1 << S1) | (1 << S2);
    
    timer1_init();

    while (1) {
        displayTime();
    }
}
\end{lstlisting}
- **Configures I/O pins** for **BCD outputs** and **digit control**.
- **Starts Timer1** for automatic timekeeping.
- **Continuously updates the display** in an infinite loop.
