\documentclass[a4paper,12pt]{article}
\let\negmedspace\undefined
\let\negthickspace\undefined
\usepackage{cite}
\usepackage{amsmath,amssymb,amsfonts,amsthm}
\usepackage{amsmath}
\usepackage{algorithmic}
\usepackage{graphicx}
\usepackage{textcomp}
\usepackage{xcolor}
\usepackage{txfonts}
\usepackage{listings}
\usepackage{multicol}
\usepackage{enumitem}
\usepackage{mathtools}
\usepackage{gensymb}
\usepackage{comment}
\usepackage[breaklinks=true]{hyperref}
\usepackage{tkz-euclide} 
\usepackage{listings}
\usepackage{gvv}                                        
\usepackage[latin1]{inputenc}                                
\usepackage{color}                                            
\usepackage{array}                                            
\usepackage{longtable}                                       
\usepackage{calc}                                             
\usepackage{multirow}                                         
\usepackage{hhline}                                           
\usepackage{ifthen}                                           
\usepackage{lscape}
\usepackage{tabularx}
\usepackage{array}
\usepackage{float}
\usepackage{circuitikz}
\newtheorem{theorem}{Theorem}[section]
\newtheorem{problem}{Problem}
\newtheorem{proposition}{Proposition}[section]
\newtheorem{lemma}{Lemma}[section]
\newtheorem{corollary}[theorem]{Corollary}
\newtheorem{example}{Example}[section]
\newtheorem{definition}[problem]{Definition}
\newcommand{\BEQA}{\begin{eqnarray}}
\newcommand{\EEQA}{\end{eqnarray}}
\newcommand{\define}{\stackrel{\triangle}{=}}
\theoremstyle{remark}
\newtheorem{rem}{Remark}
\usepackage{graphicx}
\usepackage{amsmath}
\usepackage{hyperref}
\usepackage{listings}
\usepackage{float}
\usepackage{xcolor}

\definecolor{codegreen}{rgb}{0,0.6,0}
\definecolor{codegray}{rgb}{0.5,0.5,0.5}
\definecolor{codepurple}{rgb}{0.58,0,0.82}
\definecolor{backcolour}{rgb}{0.95,0.95,0.92}

\lstdefinestyle{mystyle}{
    backgroundcolor=\color{backcolour},   
    commentstyle=\color{codegreen},
    keywordstyle=\color{magenta},
    numberstyle=\tiny\color{codegray},
    stringstyle=\color{codepurple},
    basicstyle=\ttfamily\footnotesize,
    breakatwhitespace=false,         
    breaklines=true,                 
    captionpos=b,                    
    keepspaces=true,                 
    numbers=left,                    
    numbersep=5pt,                  
    showspaces=false,                
    showstringspaces=false,
    showtabs=false,                  
    tabsize=2
}

\lstset{style=mystyle}

\title{\textbf{Digital Clock}}
\author{Ellanti Rohith - EE24BTECH11020}

\begin{document}
\date{}

\maketitle

\section*{Aim}
This project is on the design and implementation of a digital clock using an \textbf{Arduino}, a \textbf{7447 BCD to 7-segment decoder}, six \textbf{7-segment displays}, and multiple \textbf{push buttons}. 

\section{Introduction}
 This project demonstrates the working of a \textbf{multiplexed digital clock} with hour, minute, and second displays, updated every second. The implementation utilizes:
\begin{itemize}
    \item \textbf{Arduino Uno} as the main processing unit.
    \item \textbf{7447 BCD to 7-segment decoder} for numerical display.
    \item \textbf{Six 7-segment displays} to show hours, minutes, and seconds.
    \item \textbf{Push buttons} for manual adjustments (Hour, Minute, and Reset).
\end{itemize}

\section{Theory and Background}
Since an Arduino has a limited number of output pins, controlling multiple 7-segment displays directly is inefficient. Instead, \textbf{multiplexing} is used.

\subsection{How Multiplexing Works}
\begin{itemize}
    \item The Arduino \textbf{activates one display at a time} while sending the corresponding digit data.
    \item It \textbf{cycles through all six displays rapidly} (about 5ms per digit).
    \item All digits appear continuously lit.
    \item This allows \textbf{six displays to be controlled using only four BCD lines and six control lines}.
\end{itemize}

\subsection{Multiplexing Process in the Code}
\begin{enumerate}
    \item The time is split into six digits: two for hours, two for minutes, and two for seconds.
    \item The Arduino outputs each digit's BCD value to the 7447 decoder.
    \item The corresponding 7-segment display is enabled while others are turned off.
    \item After a brief delay (5ms), the next display is activated.
    \item This cycle repeats continuously, creating a seamless time display.
\end{enumerate}

\subsection{7447 BCD to 7-Segment Decoder}
The \textbf{7447 decoder} takes a \textbf{4-bit binary input (BCD)} and converts it into signals for a common-anode 7-segment display. This reduces the number of required Arduino pins.

\section{Materials and Components}
\begin{enumerate}
    \item \textbf{Arduino Uno}
    \item \textbf{7447 BCD to 7-segment decoder}
    \item \textbf{Six 7-segment displays} (Common Anode)
    \item \textbf{Three push buttons} (Hour, Minute, Reset)
    \item \textbf{Resistors} 
    \item \textbf{Breadboard and jumper wires}
\end{enumerate}

\section{Circuit Design and Wiring}
\subsection{Circuit Connections}
The table below details the wiring of the components:

\begin{table}[H]
    \centering
    \begin{tabular}{|l|l|p{8cm}|}
        \hline
        \textbf{Component} & \textbf{Arduino Pin} & \textbf{Description} \\
        \hline
        BCD Input A        & 2                   & Connects to the 7447 BCD to 7-segment decoder input A \\
        BCD Input B        & 3                   & Connects to the 7447 BCD to 7-segment decoder input B \\
        BCD Input C        & 4                   & Connects to the 7447 BCD to 7-segment decoder input C \\
        BCD Input D        & 5                   & Connects to the 7447 BCD to 7-segment decoder input D \\
        7-Segment Digit 1  & 6                   & Controls the first digit of the display (Tens place of Hours) \\
        7-Segment Digit 2  & 7                   & Controls the second digit of the display (Units place of Hours) \\
        7-Segment Digit 3  & 8                   & Controls the third digit of the display (Tens place of Minutes) \\
        7-Segment Digit 4  & 9                   & Controls the fourth digit of the display (Units place of Minutes) \\
        7-Segment Digit 5  & 10                  & Controls the fifth digit of the display (Tens place of Seconds) \\
        7-Segment Digit 6  & 11                  & Controls the sixth digit of the display (Units place of Seconds) \\
        Hour Button        & 12                  & Push button to manually increase the hours value \\
        Minute Button      & 13                  & Push button to manually increase the minutes value \\
        Reset Button       & A1                  & Push button to reset the clock to 12:00:00 \\
        \hline
    \end{tabular}
\end{table}
\section{Software Implementation}
The Arduino code below manages the clock's operation:

\begin{lstlisting}[language=C++, caption= 'Code for Digital Clock']
#define F_CPU 16000000UL
#include <avr/io.h>
#include <util/delay.h>
#include <avr/interrupt.h>

#define BCD_PORT PORTD
#define BCD_DDR DDRD
#define BCD_MASK 0b00111100  // PD2 to PD5

#define COMMON_PORT PORTC
#define COMMON_DDR DDRC

#define MODE_BUTTON PB0 // Switch between Clock, Timer, and Stopwatch
#define STOPWATCH_BUTTON PB1 // Start/Stop Stopwatch and Timer

volatile int seconds = 0, minutes = 30, hours = 15;
volatile int timer_seconds = 0, timer_minutes = 0, timer_hours = 0;
volatile int stopwatch_seconds = 0, stopwatch_minutes = 0, stopwatch_hours = 0;
volatile int mode = 0; // 0 = Clock, 1 = Timer, 2 = Stopwatch
volatile int stopwatch_running = 0; // 1 = Running, 0 = Stopped

void setup() {
    // Set BCD display pins (PD2-PD5) as output
    BCD_DDR |= BCD_MASK;
    BCD_PORT &= ~BCD_MASK;

    // Set digit selector pins (PORTC) as output
    COMMON_DDR = 0xFF;
    COMMON_PORT = 0x00;

    // Enable pull-up resistors for buttons
    PORTD |= (1 << PD6) | (1 << PD7);
    PORTB |= (1 << MODE_BUTTON) | (1 << STOPWATCH_BUTTON);

    // Timer1 Setup: CTC Mode, 1-second interval
    TCCR1B |= (1 << WGM12) | (1 << CS12) | (1 << CS10);
    OCR1A = 15625; // 1-second interrupt
    TIMSK1 |= (1 << OCIE1A);

    // Debug LED on PC7 (Bit 7 of PORTC) to check if ISR is running
    DDRC |= (1 << 7);  // Set PC7 as output
    PORTC &= ~(1 << 7); // Initially turn it off

    sei(); // Enable global interrupts
}

ISR(TIMER1_COMPA_vect) {
    PORTC ^= (1 << 7); // Toggle PC7 to check ISR is running

    // Clock Mode Updates
    if (mode == 0) {
        seconds++;
        if (seconds == 60) {
            seconds = 0;
            minutes++;
            if (minutes == 60) {
                minutes = 0;
                hours = (hours + 1) % 24;
            }
        }
    }

    // Timer Countdown (only when running)
    if (mode == 1 && stopwatch_running) {  
        if (timer_seconds > 0  timer_minutes > 0  timer_hours > 0) {
            if (timer_seconds == 0) {
                if (timer_minutes > 0) {
                    timer_minutes--;
                    timer_seconds = 59;
                } else if (timer_hours > 0) {
                    timer_hours--;
                    timer_minutes = 59;
                    timer_seconds = 59;
                }
            } else {
                timer_seconds--;
            }
        }
    }

    // Stopwatch Increment
    if (mode == 2 && stopwatch_running) {
        stopwatch_seconds++;
        if (stopwatch_seconds == 60) {
            stopwatch_seconds = 0;
            stopwatch_minutes++;
            if (stopwatch_minutes == 60) {
                stopwatch_minutes = 0;
                stopwatch_hours = (stopwatch_hours + 1) % 24;
            }
        }
    }
}

void displayTime();
void setBCD(int value);
void checkButtons();

int main() {
    setup();
    while (1) {
        checkButtons();
        displayTime();
    }
}

// Function to display time on a 6-digit 7-segment display
void displayTime() {
    int digits[6];

    if (mode == 0) { // Clock Mode
        digits[0] = hours / 10;
        digits[1] = hours % 10;
        digits[2] = minutes / 10;
        digits[3] = minutes % 10;
        digits[4] = seconds / 10;
        digits[5] = seconds % 10;
    } else if (mode == 1) { // Timer Mode
        digits[0] = timer_hours / 10;
        digits[1] = timer_hours % 10;
        digits[2] = timer_minutes / 10;
        digits[3] = timer_minutes % 10;
        digits[4] = timer_seconds / 10;
        digits[5] = timer_seconds % 10;
    } else { // Stopwatch Mode
        digits[0] = stopwatch_hours / 10;
        digits[1] = stopwatch_hours % 10;
        digits[2] = stopwatch_minutes / 10;
        digits[3] = stopwatch_minutes % 10;
        digits[4] = stopwatch_seconds / 10;
        digits[5] = stopwatch_seconds % 10;
    }

    // Multiplex 7-segment display
    for (int i = 0; i < 6; i++) {
        setBCD(digits[i]); // Send the BCD value first
        COMMON_PORT = (1 << i); // Enable the corresponding digit
        _delay_us(500); // Short delay for smooth display
    }
}
// Function to set BCD output for 7-segment display
void setBCD(int value) {
    BCD_PORT = (BCD_PORT & ~BCD_MASK) | ((value << 2) & BCD_MASK);
}

// Function to check button inputs and update mode/settings
void checkButtons() {
    if (!(PIND & (1 << PD6))) {
        _delay_ms(50);
        if (!(PIND & (1 << PD6))) {
            if (mode == 0) {
                hours = (hours + 1) % 24;
                seconds = 0;
            } else if (mode == 1) {
                timer_hours = (timer_hours + 1) % 24;
                seconds = 0;
            }
            while (!(PIND & (1 << PD6))); // Wait for release
        }
    }

    if (!(PIND & (1 << PD7))) {
        _delay_ms(50);
        if (!(PIND & (1 << PD7))) {
            if (mode == 0) {
                minutes = (minutes + 1) % 60;
                seconds = 0;
            } else if (mode == 1) {
                timer_minutes = (timer_minutes + 1) % 60;
                seconds = 0;
            }
            while (!(PIND & (1 << PD7))); // Wait for release
        }
    }

    if (!(PINB & (1 << MODE_BUTTON))) {
        _delay_ms(50);
        if (!(PINB & (1 << MODE_BUTTON))) {
            mode = (mode + 1) % 3; // Cycle through Clock, Timer, and Stopwatch
            while (!(PINB & (1 << MODE_BUTTON))); // Wait for release
        }
    }

    // Modified section: Stopwatch button controls both Timer and Stopwatch
    if (!(PINB & (1 << STOPWATCH_BUTTON))) {
        _delay_ms(50);
        if (!(PINB & (1 << STOPWATCH_BUTTON))) {
            if (mode == 2) {  // Toggle Stopwatch running
                stopwatch_running = !stopwatch_running;
            } else if (mode == 1) {  // Toggle Timer running
                stopwatch_running = !stopwatch_running;  // Reuse the same flag
            }
            while (!(PINB & (1 << STOPWATCH_BUTTON))) {
                _delay_ms(10);
            }
        }
    }
}
\end{lstlisting}



\section{Conclusion}
The project successfully demonstrated a digital clock using Arduino, 7447 decoder, and 7-segment displays. The clock functions accurately with manual control options, showcasing \textbf{real-time updating, display multiplexing}.
\end{document}
