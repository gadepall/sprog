\documentclass[12pt]{article}
\usepackage{background}
\usepackage{graphicx}
\usepackage[margin=1in]{geometry}
\usepackage{setspace}
\usepackage{hyperref}
\usepackage{xcolor}
\usepackage{tikz}
\usepackage{listings}
\usepackage{xcolor}
\usepackage{float}
\usepackage{subcaption}
\usepackage{circuitikz}
\usepackage[justification=centering]{caption}
\lstset{
  language=Python,
  backgroundcolor=\color{black!5},   % light gray background
  basicstyle=\ttfamily\small,         % Monospaced font for code
  breaklines=true,                    % Line wrapping
  keywordstyle=\color{blue},           % Keywords in blue
  commentstyle=\color{green},         % Comments in green
  stringstyle=\color{red},            % Strings in red
  identifierstyle=\color{black},      % Identifiers in black
  morekeywords={def,class}, 
           % Function and class names in bold
  morekeywords={import, as},      % Add extra keywords to be highlighted
                      % Space between line numbers and code
  frame=single,                       % Single line frame around the code
  rulecolor=\color{black},            % Frame color
  tabsize=4,                          % Number of spaces per tab
  showstringspaces=false              % Don't underline spaces in strings
}
\backgroundsetup{
  scale=0.5,
  color=black,
  opacity=0,  % Reduce opacity further (e.g., 0.05)
  angle=0,
  position=current page.center,
  vshift=-5cm,
  hshift=0cm,
  contents={\includegraphics[width=\paperwidth, height=\paperheight]{figs/logo.jpg}}
}


% Fix section numbering
\setcounter{secnumdepth}{0}

\begin{document}
\begin{titlepage}
    \centering
    {\Huge \bfseries  Digital Clock Project \par}
    \vspace{1cm}
    \includegraphics[width=5cm]{figs/logo.jpg} % Replace with your logo file
    \vspace{1cm}
    
    {\Large \bfseries Project Report \par}
    \vspace{0.5cm}
    
    {\large EE1003: Scientific Programming \par}
    \vspace{2cm}
    \begin{tabular}{ll}
    \textbf{S. Rohith Sai} & \textbf{EE24BTECH11061}  \end{tabular}
\vspace{1cm}
\end{titlepage}
\newpage
\tableofcontents
\newpage

\section{\textbf{Digital Clock}}
\subsection{Design Objectives}
The digital clock displays time accurately in 24-hour format (HH:MM:SS).

\subsection{Hardware Components}
The digital clock implementation utilizes the following components:

\begin{itemize}
    \item \textbf{Microcontroller:} AVR microcontroller for processing, time calculations, and display control
    \item \textbf{7447 BCD-to-Seven-Segment Decoder:} Converts Binary Coded Decimal (BCD) inputs to drive seven-segment displays
    \item \textbf{Six 7-Segment Displays:} Shows hours, minutes, and seconds in 24-hour format (HH:MM:SS)
    \item \textbf{Multiplexing Circuit:} Controls all six displays using a single set of segment outputs
\end{itemize}

\subsection{Connection Details}
\subsubsection{BCD Input Connections}
\begin{itemize}
    \item 7447 Input A (Pin 7) $\rightarrow$ AVR Pin D2
    \item 7447 Input B (Pin 1) $\rightarrow$ AVR Pin D3
    \item 7447 Input C (Pin 2) $\rightarrow$ AVR Pin D4
    \item 7447 Input D (Pin 6) $\rightarrow$ AVR Pin D5
\end{itemize}
    
\subsubsection{7-Segment Display Controls}
\begin{itemize}
    \item Segment 'a' (Pin 13) $\rightarrow$ Connected to segment 'a' of all displays
    \item Segment 'b' (Pin 12) $\rightarrow$ Connected to segment 'b' of all displays
    \item Segment 'c' (Pin 11) $\rightarrow$ Connected to segment 'c' of all displays
    \item Segment 'd' (Pin 10) $\rightarrow$ Connected to segment 'd' of all displays
    \item Segment 'e' (Pin 9) $\rightarrow$ Connected to segment 'e' of all displays
    \item Segment 'f' (Pin 15) $\rightarrow$ Connected to segment 'f' of all displays
    \item Segment 'g' (Pin 14) $\rightarrow$ Connected to segment 'g' of all displays
\end{itemize}
    
\subsubsection{Display Selection}
\begin{itemize}
    \item Tens of Hours (TH) $\rightarrow$ AVR Pin D6
    \item Units of Hours (UH) $\rightarrow$ AVR Pin D7
    \item Tens of Minutes (TM) $\rightarrow$ AVR Pin D8
    \item Units of Minutes (UM) $\rightarrow$ AVR Pin D9
    \item Tens of Seconds (TS) $\rightarrow$ AVR Pin D10
    \item Units of Seconds (US) $\rightarrow$ AVR Pin D11
\end{itemize}

The multiplexing technique activates only one display at a time. Persistence of vision makes it appear as if all displays are simultaneously illuminated, reducing required microcontroller pins while maintaining display quality.

\subsection{User Interface}
The digital clock incorporates buttons connected to the AVR microcontroller's analog input pins:

\begin{itemize}
    \item \textbf{A0 (PC0):} Increase Hours (H+)
    \item \textbf{A1 (PC1):} Decrease Hours (H-)
    \item \textbf{A2 (PC2):} Increase Minutes (M+)
    \item \textbf{A3 (PC3):} Decrease Minutes (M-)
    \item \textbf{A4 (PC4):} Reset Clock (00:00:00)
    \item \textbf{A5 (PC5):} Start/Stop Clock
\end{itemize}

All buttons use software debouncing and pull-up resistors for reliable operation.

\subsection{Testing and Validation}
\subsubsection{Functional Testing}
\begin{itemize}
    \item \textbf{Time Accuracy:} Long-term tests against reference clock
    \item \textbf{Button Response:} Verification of reliable debounced operation
    \item \textbf{Display Quality:} Assessment of readability and refresh rate
    \item \textbf{Edge Case Handling:} Testing time rollover at 24-hour boundaries
\end{itemize}

\subsubsection{Reliability Testing}
\begin{itemize}
    \item \textbf{Long-term Operation:} Continuous running for extended periods
    \item \textbf{Button Durability:} Repeated button press testing
    \item \textbf{Power Cycle:} Multiple power-on/power-off cycles to ensure consistent operation
\end{itemize}

\textbf{Note}\\
I modified \textbf{J. Kedarananda's (EE24BTECH11030)} code according to my connections and button functions.
\end{document}