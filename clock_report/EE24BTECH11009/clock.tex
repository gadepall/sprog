\documentclass[journal]{IEEEtran}
\usepackage[a5paper, margin=10mm, onecolumn]{geometry}
%\usepackage{lmodern} % Ensure lmodern is loaded for pdflatex
\usepackage{tfrupee} % Include tfrupee package

\setlength{\headheight}{1cm} % Set the height of the header box
\setlength{\headsep}{0mm}     % Set the distance between the header box and the top of the text

\usepackage{gvv-book}
\usepackage{gvv}
\usepackage{cite}
\usepackage{amsmath,amssymb,amsfonts,amsthm}
\usepackage{algorithmic}
\usepackage{graphicx}
\usepackage{textcomp}
\usepackage{xcolor}
\usepackage{txfonts}
\usepackage{listings}
\usepackage{enumitem}
\usepackage{mathtools}
\usepackage{gensymb}
\usepackage{comment}
\usepackage[breaklinks=true]{hyperref}
\usepackage{tkz-euclide}
\usepackage{listings}
% \usepackage{gvv}                                        
\def\inputGnumericTable{}                                 
\usepackage[latin1]{inputenc}                                
\usepackage{color}                                            
\usepackage{array}                                            
\usepackage{longtable}                                       
\usepackage{calc}                                             
\usepackage{multirow}                                         
\usepackage{hhline}                                           
\usepackage{ifthen}                                           
\usepackage{lscape}
\begin{document}

\bibliographystyle{IEEEtran}
\vspace{3cm}

\title{CLOCK}
\author{EE24BTECH11009 - Mokshith Kumar}
% \maketitle
% \newpage
% \bigskip
{\let\newpage\relax\maketitle}

\renewcommand{\thefigure}{\theenumi}
\renewcommand{\thetable}{\theenumi}
\setlength{\intextsep}{10pt} % Space between text and floats


\numberwithin{equation}{enumi}
\numberwithin{figure}{enumi}
\renewcommand{\thetable}{\theenumi}
\tableofcontents
\newpage

\section{Introduction}
In this guide, we will build a 6-digit clock (HH:MM:SS) using an Arduino, a 74447 BCD to 7-segment decoder, and a multiplexing technique to efficiently display the time. This project helps you learn about timekeeping, display control, and embedded programming.

\section{Components Required}
Before we start, we need the following components:
\begin{itemize}
\item Arduino Uno
\item 74447 BCD to 7-Segment Decoder
\item 6 x Common-Anode 7-Segment Displays
\item Breadboard and Jumper Wires
\item Resistors ($220 \ohm$)
\item $5V$Power Supply
\end{itemize}

\section{Wiring the Circuit}
 \subsection{Connections of 7447}
The table below shows the connections between the 7447 decoder and the Arduino Uno:

\begin{table}[H]
    \centering
    \caption{7447 to Arduino Pin Connections}
    \begin{tabular}{|c|c|}
        \hline
        \textbf{7447 Pin} & \textbf{Arduino Connection} \\
        \hline
        VCC (Pin 16) & +5V \\
        GND (Pin 8) & GND \\
        A (Pin 7) & D2 (BCD input LSB) \\
        B (Pin 1) & D3 \\
        C (Pin 2) & D4 \\
        D (Pin 6) & D5 (BCD input MSB) \\
        \hline
    \end{tabular}
\end{table}
\subsection{Connections to seven segment display}
The 7447 outputs are connected to the corresponding segments of all six displays, as shown in the table below:

\begin{table}[H]
    \centering
    \caption{7447 to Display Segment Connections}
    \begin{tabular}{|c|c|}
        \hline
        \textbf{7447 Pin} & \textbf{Display Segment} \\
        \hline
        Pin 13 & Segment "a" \\
        Pin 12 & Segment "b" \\
        Pin 14 & Segment "c" \\
        Pin 15 & Segment "d" \\
        Pin 9  & Segment "e" \\
        Pin 10 & Segment "f" \\
        Pin 11 & Segment "g" \\
        \hline
    \end{tabular}
\end{table}
\subsection{connections to Arduino}
The wiring involves connecting:
 Arduino to the 74447 for BCD control as in the module, then connect 74447 to the 7-segment display segments(to each of them as if thay are in series).Common-anode pins of each display to Arduino pins. Buttons for time adjustment and mode selection(optional).\\
 Proper wiring ensures smooth operation and clear display visibility.\\
\begin{table}[H]
    \centering
    \caption{Common Anode Connections}
    \begin{tabular}{|c|c|}
        \hline
        \textbf{Display} & \textbf{Arduino Analog Pin (via 180$\Omega$ resistor)} \\
        \hline
        Display 1 & A0 \\
        Display 2 & A1 \\
        Display 3 & A2 \\
        Display 4 & A3 \\
        Display 5 & A4 \\
        Display 6 & A5 \\
        \hline
    \end{tabular}
\end{table}
\section{Programming the Arduino}
\subsection{overview}
The Arduino controls the clock using:
A Timer Interrupt to update time every second. Multiplexing to switch between digits quickly. Button Handling to adjust time and change modes. A simple loop continuously updates the display using BCD encoding and digit control.
\subsection{code}
\begin{lstlisting}[language = C]
#define F_CPU 16000000UL
#include <avr/io.h>
#include <util/delay.h>
#include <avr/interrupt.h>

#define BCD_PORT PORTD
#define BCD_DDR DDRD
#define BCD_MASK 0b00111100  // PD2 to PD5

#define COMMON_PORT PORTC
#define COMMON_DDR DDRC

#define MODE_BUTTON PB0 // Switch between Clock, Timer, and Stopwatch
#define STOPWATCH_BUTTON PB1 // Start/Stop Stopwatch

volatile int seconds = 0, minutes = 30, hours = 15;
volatile int timer_seconds = 0, timer_minutes = 0, timer_hours = 0;
volatile int stopwatch_seconds = 0, stopwatch_minutes = 0, stopwatch_hours = 0;
volatile int mode = 0; // 0 = Clock, 1 = Timer, 2 = Stopwatch
volatile int stopwatch_running = 0; // 1 = Running, 0 = Stopped

void setup() {
    // Set BCD display pins (PD2-PD5) as output
    BCD_DDR |= BCD_MASK;
    BCD_PORT &= ~BCD_MASK;

    // Set digit selector pins (PORTC) as output
    COMMON_DDR = 0xFF;
    COMMON_PORT = 0x00;

    // Enable pull-up resistors for buttons
    PORTD |= (1 << PD6) | (1 << PD7);
    PORTB |= (1 << MODE_BUTTON) | (1 << STOPWATCH_BUTTON);

    // Timer1 Setup: CTC Mode, 1-second interval
    TCCR1B |= (1 << WGM12) | (1 << CS12) | (1 << CS10);
    OCR1A = 15625; // 1-second interrupt
    TIMSK1 |= (1 << OCIE1A);

    // Debug LED on PC7 (Bit 7 of PORTC) to check if ISR is running
    DDRC |= (1 << 7);  // Set PC7 as output
    PORTC &= ~(1 << 7); // Initially turn it off

    sei(); // Enable global interrupts
}


void displayTime();
void setBCD(int value);
void checkButtons();

int main() {
    setup();
    while (1) {
        checkButtons();
        displayTime();
    }
}

// Function to display time on a 6-digit 7-segment display
void displayTime() {
    int digits[6];

    if (mode == 0) { // Clock Mode
        digits[0] = hours / 10;
        digits[1] = hours % 10;
        digits[2] = minutes / 10;
        digits[3] = minutes % 10;
        digits[4] = seconds / 10;
        digits[5] = seconds % 10;
    } else if (mode == 1) { // Timer Mode
        digits[0] = timer_hours / 10;
        digits[1] = timer_hours % 10;
        digits[2] = timer_minutes / 10;
        digits[3] = timer_minutes % 10;
        digits[4] = timer_seconds / 10;
        digits[5] = timer_seconds % 10;
    } else { // Stopwatch Mode
        digits[0] = stopwatch_hours / 10;
        digits[1] = stopwatch_hours % 10;
        digits[2] = stopwatch_minutes / 10;
        digits[3] = stopwatch_minutes % 10;
        digits[4] = stopwatch_seconds / 10;
        digits[5] = stopwatch_seconds % 10;
    }

    // Multiplex 7-segment display
    for (int i = 0; i < 6; i++) {
        setBCD(digits[i]); // Send the BCD value first
        COMMON_PORT = (1 << i); // Enable the corresponding digit
        _delay_us(500); // Short delay for smooth display
    }
}


// Function to set BCD output for 7-segment display
void setBCD(int value) {
    BCD_PORT = (BCD_PORT & ~BCD_MASK) | ((value << 2) & BCD_MASK);
}

// Function to check button inputs and update mode/settings
void checkButtons() {
    if (!(PIND & (1 << PD6))) {
        _delay_ms(50);
        if (!(PIND & (1 << PD6))) {
            if (mode == 0) {
                hours = (hours + 1) % 24;
                seconds = 0;
            } else if (mode == 1) {
                timer_hours = (timer_hours + 1) % 24;
                seconds = 0;
            }
            while (!(PIND & (1 << PD6))); // Wait for release
        }
    }

    if (!(PIND & (1 << PD7))) {
        _delay_ms(50);
        if (!(PIND & (1 << PD7))) {
            if (mode == 0) {
                minutes = (minutes + 1) % 60;
                seconds = 0;
            } else if (mode == 1) {
                timer_minutes = (timer_minutes + 1) % 60;
                seconds = 0;
            }
            while (!(PIND & (1 << PD7))); // Wait for release
        }
    }

    if (!(PINB & (1 << MODE_BUTTON))) {
        _delay_ms(50);
        if (!(PINB & (1 << MODE_BUTTON))) {
            mode = (mode + 1) % 3; // Cycle through Clock, Timer, and Stopwatch
            while (!(PINB & (1 << MODE_BUTTON))); // Wait for release
        }
    }
}
\end{lstlisting}
\section{Implementation}
\subsection{How Multiplexing Works}
In principle of multiplexing only one display is active at a time. The microcontroller(Arduino) sends the appropriate digit to the common data bus. A control signal activates the desired display. This process repeats quickly (~1ms per digit), making all digits appear simultaneously to the human eye. like this we can create an illusion of the displays lighted continuously.

\section{Testing and Calibration}
After uploading the code, verify that:
\begin{itemize}
\item The time increments correctly.
\item The buttons adjust the time as expected.
\item The display is stable and flicker-free.
\end{itemize}
Adjust delays if flickering occurs, and check wiring if digits display incorrectly.

 
\end{document}
