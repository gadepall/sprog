\documentclass[a4paper,12pt]{article}
\usepackage{cite}
\usepackage{amsmath, amssymb, amsfonts, amsthm}
\usepackage{graphicx}
\usepackage{textcomp}
\usepackage{xcolor}
\usepackage{listings}
\usepackage{multicol}
\usepackage{enumitem}
\usepackage{mathtools}
\usepackage{hyperref}
\usepackage{float}
\usepackage{circuitikz}

\definecolor{codegreen}{rgb}{0,0.6,0}
\definecolor{codegray}{rgb}{0.5,0.5,0.5}
\definecolor{codepurple}{rgb}{0.58,0,0.82}
\definecolor{backcolour}{rgb}{0.95,0.95,0.92}

\lstdefinestyle{customstyle}{
    backgroundcolor=\color{backcolour},   
    commentstyle=\color{codegreen},
    keywordstyle=\color{magenta},
    numberstyle=\tiny\color{codegray},
    stringstyle=\color{codepurple},
    basicstyle=\ttfamily\footnotesize,
    breaklines=true,                 
    captionpos=b,                    
    keepspaces=true,                 
    numbers=left,                    
    numbersep=5pt,                  
    showspaces=false,                
    showstringspaces=false,
    showtabs=false,                  
    tabsize=2
}

\lstset{style=customstyle}

\title{\textbf{Digital Clock Using Arduino and 7-Segment Displays}}
\author{DHWANITH M DODDAHUNDI - EE24BTECH11016}
\date{}

\begin{document}

\maketitle

\section*{Objective}
The purpose of this project is to design and build a digital clock using:
\begin{itemize}
    \item \textbf{Arduino Uno} as the core processing unit.
    \item \textbf{7447 BCD to 7-segment decoder} to simplify the control of the 7-segment displays.
    \item \textbf{Six 7-segment displays} to represent hours, minutes, and seconds.
    \item \textbf{Push buttons} for manual adjustments.
\end{itemize}

\section{Overview}
The digital clock is based on a multiplexing technique to reduce the required output pins from the Arduino. The time is displayed across six 7-segment displays, which are controlled by a BCD to 7-segment decoder.

\section{Theory and Design Principles}

\subsection{Multiplexing Concept}
Since the Arduino has a limited number of output pins, directly driving six 7-segment displays would be inefficient. Instead, multiplexing is used:
\begin{itemize}
    \item The Arduino activates one display at a time, while sending the digit data.
    \item This creates the illusion that all digits are continuously illuminated.
    \item This technique reduces the required pins by sharing BCD lines across multiple displays.
\end{itemize}

\subsection{7447 BCD to 7-Segment Decoder}
The 7447 decoder translates a 4-bit binary-coded decimal (BCD) input into signals for controlling a 7-segment display. This minimizes the number of Arduino pins required for display control.

\subsection{Time Management with Arduino}
The clock uses the \texttt{millis()} function to manage timekeeping, which tracks milliseconds since the program started. By counting 1000 milliseconds, the clock increments the seconds.

\section{Components and Circuit Connections}
\subsection{Hardware Requirements}
\begin{itemize}
    \item Arduino Uno
    \item 7447 BCD to 7-segment decoder
    \item Six 7-segment displays
    \item Three push buttons (Hour, Minute, and Reset)
    \item Resistors
    \item Breadboard and jumper wires
\end{itemize}

\subsection{Wiring Details}
The table below shows the connections between the Arduino and the components:

\begin{table}[H]
\centering
\begin{tabular}{|l|l|p{8cm}|}
    \hline
    \textbf{Component} & \textbf{Arduino Pin} & \textbf{Purpose} \\
    \hline
    BCD Input A        & 2                   & 7447 BCD input A \\
    BCD Input B        & 3                   & 7447 BCD input B \\
    BCD Input C        & 4                   & 7447 BCD input C \\
    BCD Input D        & 5                   & 7447 BCD input D \\
    7-Segment Digit 1  & 6                   & Tens place of Hours \\
    7-Segment Digit 2  & 7                   & Units place of Hours \\
    7-Segment Digit 3  & 8                   & Tens place of Minutes \\
    7-Segment Digit 4  & 9                   & Units place of Minutes \\
    7-Segment Digit 5  & 10                  & Tens place of Seconds \\
    -Segment Digit 6  & 11                  & Units place of Seconds \\
    Hour Button        & 12                  & Increases the hour value \\
    Minute Button      & 13                  & Increases the minute value \\
    Reset Button       & A0                  & Resets the clock \\
    \hline
\end{tabular}
\end{table}

\section{Arduino Code Implementation}
The following Arduino code manages the clock's operation, including time updates, display multiplexing, and button handling.

\begin{lstlisting}[language=C++, caption=Arduino Code for Digital Clock]
// Digital Clock with Arduino and 7-Segment Displays
#include <avr/io.h>
#include <util/delay.h>
#include <avr/interrupt.h>

#define BCD_PORT PORTD
#define BCD_DDR DDRD
#define BCD_MASK 0b00111100  // PD2 to PD5

#define COMMON_PORT PORTC
#define COMMON_DDR DDRC

#define MODE_BUTTON PB0 // Switch between Clock, Timer, and Stopwatch
#define STOPWATCH_BUTTON PB1 // Start/Stop Stopwatch and Timer

volatile int seconds = 0, minutes = 30, hours = 15;
volatile int timer_seconds = 0, timer_minutes = 0, timer_hours = 0;
volatile int stopwatch_seconds = 0, stopwatch_minutes = 0, stopwatch_hours = 0;
volatile int mode = 0; // 0 = Clock, 1 = Timer, 2 = Stopwatch
volatile int stopwatch_running = 0; // 1 = Running, 0 = Stopped

void setup() {
    // Set BCD display pins (PD2-PD5) as output
    BCD_DDR |= BCD_MASK;
    BCD_PORT &= ~BCD_MASK;

    // Set digit selector pins (PORTC) as output
    COMMON_DDR = 0xFF;
    COMMON_PORT = 0x00;

    // Enable pull-up resistors for buttons
    PORTD |= (1 << PD6) | (1 << PD7);
    PORTB |= (1 << MODE_BUTTON) | (1 << STOPWATCH_BUTTON);

    // Timer1 Setup: CTC Mode, 1-second interval
    TCCR1B |= (1 << WGM12) | (1 << CS12) | (1 << CS10);
    OCR1A = 15625; // 1-second interrupt
    TIMSK1 |= (1 << OCIE1A);

    // Debug LED on PC7 (Bit 7 of PORTC) to check if ISR is running
    DDRC |= (1 << 7);  // Set PC7 as output
    PORTC &= ~(1 << 7); // Initially turn it off

    sei(); // Enable global interrupts
}

ISR(TIMER1_COMPA_vect) {
    PORTC ^= (1 << 7); // Toggle PC7 to check ISR is running

    // Clock Mode Updates
    if (mode == 0) {
        seconds++;
        if (seconds == 60) {
            seconds = 0;
            minutes++;
            if (minutes == 60) {
                minutes = 0;
                hours = (hours + 1) % 24;
            }
        }
    }

    // Timer Countdown (only when running)
    if (mode == 1 && stopwatch_running) {  
        if (timer_seconds > 0  timer_minutes > 0  timer_hours > 0) {
            if (timer_seconds == 0) {
                if (timer_minutes > 0) {
                    timer_minutes--;
                    timer_seconds = 59;
                } else if (timer_hours > 0) {
                    timer_hours--;
                    timer_minutes = 59;
                    timer_seconds = 59;
                }
            } else {
                timer_seconds--;
            }
        }
    }

    // Stopwatch Increment
    if (mode == 2 && stopwatch_running) {
        stopwatch_seconds++;
        if (stopwatch_seconds == 60) {
            stopwatch_seconds = 0;
            stopwatch_minutes++;
            if (stopwatch_minutes == 60) {
                stopwatch_minutes = 0;
                stopwatch_hours = (stopwatch_hours + 1) % 24;
            }
        }
    }
}

void displayTime();
void setBCD(int value);
void checkButtons();

int main() {
    setup();
    while (1) {
        checkButtons();
        displayTime();
    }
}

// Function to display time on a 6-digit 7-segment display
void displayTime() {
    int digits[6];

    if (mode == 0) { // Clock Mode
        digits[0] = hours / 10;
        digits[1] = hours % 10;
        digits[2] = minutes / 10;
        digits[3] = minutes % 10;
        digits[4] = seconds / 10;
        digits[5] = seconds % 10;
    } else if (mode == 1) { // Timer Mode
        digits[0] = timer_hours / 10;
        digits[1] = timer_hours % 10;
        digits[2] = timer_minutes / 10;
        digits[3] = timer_minutes % 10;
        digits[4] = timer_seconds / 10;
        digits[5] = timer_seconds % 10;
    } else { // Stopwatch Mode
        digits[0] = stopwatch_hours / 10;
        digits[1] = stopwatch_hours % 10;
        digits[2] = stopwatch_minutes / 10;
        digits[3] = stopwatch_minutes % 10;
        digits[4] = stopwatch_seconds / 10;
        digits[5] = stopwatch_seconds % 10;
    }

    // Multiplex 7-segment display
    for (int i = 0; i < 6; i++) {
        setBCD(digits[i]); // Send the BCD value first
        COMMON_PORT = (1 << i); // Enable the corresponding digit
        _delay_us(500); // Short delay for smooth display
    }
}
// Function to set BCD output for 7-segment display
void setBCD(int value) {
    BCD_PORT = (BCD_PORT & ~BCD_MASK) | ((value << 2) & BCD_MASK);
}

// Function to check button inputs and update mode/settings
void checkButtons() {
    if (!(PIND & (1 << PD6))) {
        _delay_ms(50);
        if (!(PIND & (1 << PD6))) {
            if (mode == 0) {
                hours = (hours + 1) % 24;
                seconds = 0;
            } else if (mode == 1) {
                timer_hours = (timer_hours + 1) % 24;
                seconds = 0;
            }
            while (!(PIND & (1 << PD6))); // Wait for release
        }
    }

    if (!(PIND & (1 << PD7))) {
        _delay_ms(50);
        if (!(PIND & (1 << PD7))) {
            if (mode == 0) {
                minutes = (minutes + 1) % 60;
                seconds = 0;
            } else if (mode == 1) {
                timer_minutes = (timer_minutes + 1) % 60;
                seconds = 0;
            }
            while (!(PIND & (1 << PD7))); // Wait for release
        }
    }

    if (!(PINB & (1 << MODE_BUTTON))) {
        _delay_ms(50);
        if (!(PINB & (1 << MODE_BUTTON))) {
            mode = (mode + 1) % 3; // Cycle through Clock, Timer, and Stopwatch
            while (!(PINB & (1 << MODE_BUTTON))); // Wait for release
        }
    }

    // Modified section: Stopwatch button controls both Timer and Stopwatch
    if (!(PINB & (1 << STOPWATCH_BUTTON))) {
        _delay_ms(50);
        if (!(PINB & (1 << STOPWATCH_BUTTON))) {
            if (mode == 2) {  // Toggle Stopwatch running
                stopwatch_running = !stopwatch_running;
            } else if (mode == 1) {  // Toggle Timer running
                stopwatch_running = !stopwatch_running;  // Reuse the same flag
            }
            while (!(PINB & (1 << STOPWATCH_BUTTON))) {
                _delay_ms(10);
            }
        }
    }
}
\end{lstlisting}

\section{Conclusion}
This project successfully demonstrates a functional digital clock with real-time updates and display multiplexing. The implementation uses an Arduino Uno, a 7447 decoder, and 7-segment displays, making it efficient and easy to operate.

\end{document}
7
