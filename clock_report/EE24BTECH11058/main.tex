\documentclass{article}
\usepackage{titlesec}
\usepackage{enumitem}
\usepackage{listings}
\usepackage{xcolor}

\title{\textbf{Hardware Project - Digital Clock }}
\author{P. Shiny Diavajna\\ EE24BTCH11058}
\date{\today}

\begin{document}

\maketitle

\section{Introduction}
This report details the design and implementation of a digital clock using the ATmega328p microcontroller (in Arduino) and six seven-segment displays. Timer1 interrupts are utilized for updating time, and multiplexing is used to control multiple seven-segment displays efficiently. The initial time can be manually set in the code.

\section{Components Used}
\begin{itemize}
    \item Arduino UNO
    \item Six seven-segment displays
    \item Resistors
    \item Jumper wires
    \item Breadboard
    \item Power supply
\end{itemize}

\section{Circuit Design}
\begin{itemize}
    \item Place the six seven-segment displays on the breadboard. The first two are used to display the two digits of hours, the next two for minutes, and the next two for seconds.
    \item Connect the 5V and GND pins from the Arduino to the power rails of the breadboard.
    \item Connect the COM of each display to the 5V rail through resistors.
    \item Connect the dot of each display to GND.
    \item Make connections from the Arduino pins to the seven-segment displays as per the table below.
\end{itemize}

\begin{tabular}{|c|c|}
    \hline
    \textbf{Variable} & \textbf{Description}\\
    \hline
    $P_0$ & initial principal amount\\
    \hline
    $r$ & rate of increase per year\\
    \hline
    $t$ & time in years\\
    \hline 
    $C \& C_1$ & arbitrary constants\\
    \hline
    $P$ & principal at any time $t$\\
    \hline
\end{tabular}


\section{Code}
\begin{lstlisting}[language=C, basicstyle=\ttfamily, keywordstyle=\color{blue}, commentstyle=\color{green}]
// setting cpu frequency to 16 mega hz (for atmega328p)
// atmega328p microcontroller is in the arduino
// frequency is needed for timing calculations
#define F_CPU 16000000UL

// including required libraries
// they provide access to hardware registers, interrupts, delays
#include <avr/io.h> // standard input-output functions for avr
#include <avr/interrupt.h> // interrupt handling
#include <util/delay.h> // delay functions

// array to store bit patterns for each digit
// low (0) turns on the led segment
const uint8_t digit_map[] = {
    0b00000000,  // 0
    0b11100100,  // 1
    0b10010000,  // 2
    0b11000000,  // 3
    0b01100100,  // 4
    0b01001000,  // 5
    0b00001000,  // 6
    0b11100000,  // 7
    0b00000000,  // 8
    0b01000000   // 9
};

// defining time variables
volatile uint8_t hours = 5, minutes = 11, seconds = 10;

// array to store display digits
uint8_t digits[6];
// function to calculate digit values for display
void update_digits() {
    digits[0] = hours / 10;
    digits[1] = hours % 10;
    digits[2] = minutes / 10;
    digits[3] = minutes % 10;
    digits[4] = seconds / 10;
    digits[5] = seconds % 10;
}

// function to update time at every second
void update_time() {
    seconds++;
    if (seconds >= 60) { seconds = 0; minutes++; }
    if (minutes >= 60) { minutes = 0; hours++; }
    if (hours >= 24) { hours = 0; }
    update_digits();
}

// interrupt service routine (ISR) for Timer1 compare match A
ISR(TIMER1_COMPA_vect) {
    update_time();
}

// function to display a single digit (multiplexing)
void display_digit(uint8_t display, uint8_t digit) {
    PORTB &= ~(0b00011110);
    PORTC &= ~(0b00000011);
    PORTD = digit_map[digit];
    if (digit == 0 || digit == 1 || digit == 7) {
        PORTB |= (1 << PB0);
    } else {
        PORTB &= ~(1 << PB0);
    }
    if (display < 4) {
        PORTB |= (1 << (display + 1));
    } else {
        PORTC |= (1 << (display - 4));
    }
    _delay_ms(2);
}

int main(void) {
    DDRD |= 0b11111100;
    DDRB |= (1 << PB0);
    DDRB |= (1 << PB1) | (1 << PB2) | (1 << PB3) | (1 << PB4);
    DDRC |= (1 << PC0) | (1 << PC1);
    update_digits();
    TCCR1B |= (1 << WGM12) | (1 << CS12) | (1 << CS10);
    OCR1A = 15625;
    TIMSK1 |= (1 << OCIE1A);
    sei();
    while (1) {
        for (uint8_t i = 0; i < 6; i++) {
            display_digit(i, digits[i]);
        }
    }
}
\end{lstlisting}

\section{Results}
The clock successfully displays the time in the HH:MM:SS format, with precise timing and a clear display, which can be attributed to the usage of multiplexing and Timer1 interrupts.

\section{Conclusion}
This project demonstrates the use of AVR-GCC and microcontroller timers for implementing a digital clock. The current implementation requires manual time setting during initialization.\\\\
Acknowlwdgement : Code sourced from M.Srujana EE24BTECH11042
\end{document}

