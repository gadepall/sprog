\documentclass[12pt,a4paper]{article}
\usepackage{graphicx}
\usepackage{amsmath}
\usepackage{listings}
\usepackage{xcolor}
\usepackage{caption}
\usepackage{subcaption}
\usepackage{booktabs}

\title{\textbf{Digital Clock Using AVR-GCC}}
\author{Sruthi Bijili \\EE24BTECH11060}


\begin{document}

\maketitle

\section{Objective}
To design and implement a digital clock using an Arduino board and a seven-segment display, programmed using AVR-GCC.

\section{Introduction}
This experiment demonstrates the implementation of a digital clock using an Arduino board. The timekeeping is achieved through a timer interrupt, while multiplexed seven-segment displays are used to show the time in HH:MM:SS format.

\section{Apparatus}
\begin{itemize}
\item bread board
\item Seven-segment display (Common Anode)
\item Resistors
\item Arduino UNO
\item power supply
\item Connecting Wires
\end{itemize}
\section{Connections}
\begin{itemize}
\item Place the six seven-segment displays on the breadboard. The first two are used to display the two digits of hours, the next two for minutes, and the next two for seconds.
    \item Connect the 5V and GND pins from the Arduino to the power rails of the breadboard.
    \item Connect the COM of each display to the 5V rail through resistors.
    \item Connect the dot of each display to GND.
    \item Make connections from the Arduino pins to the seven-segment displays as per the table below.
\end{itemize}
\begin{table}[H]
\centering
\caption{Connections}
\label{tab:connections}
\begin{tabular}{ccc}
\toprule
First end of jumper wire & Second end of jumper wire \\
\midrule
Arduino digital pin 0 & Push button no. 16 \\
pin 1 & Push button no.17 \\
pin 2 & LCD pin 4 \\
pin 3 & LCD pin 6 \\
pin 4 & LCD pin 11 \\
pin 5 & LCD pin 12 \\
pin 6 & LCD pin 13 \\
pin 7 & LCD pin 14 \\
pin 8 & Push button no.18 \\
pin 9 & Push button no. 19 \\
pin 10 & Push button no. 20 \\
pin 11 & Push button no. 21 \\
pin 12 & Push button no. 22 \\
pin 13 & Push button no. 23 \\
Arduino analog pin A1 & Push button no. 15 \\
pin A2 & Push button no. 14 \\
pin A3 & Push button no. 13 \\
pin A4 & Push button no. 12 \\
pin A5 & Push button no. 11 \\
pin A0 & Push buttons no. 1-10 (digit buttons) \\
LCD pin 1 & Ground \\
LCD pin 2 & 5V \\
LCD pin 15 & 5V via 1k$\Omega$ resistor \\
LCD pin 16 & Ground \\
LCD pin 3 & Ground via 1.5k$\Omega$ resistor \\
LCD pin 5 & Ground \\
LCD pin 5 & All push buttons \\
\bottomrule
\end{tabular}
\end{table}

\section{AVR Code}
\begin{lstlisting}[language=C, basicstyle=\ttfamily, keywordstyle=\color{blue}, commentstyle=\color{green}]
// setting cpu frequency to 16 mega hz (for atmega328p)
// atmega328p microcontroller is in the arduino
// frequency is needed for timing calculations
#define F_CPU 16000000UL

// including required libraries
// they provide access to hardware registers, interrupts, delays
#include <avr/io.h> // standard input-output functions for avr
#include <avr/interrupt.h> // interrupt handling
#include <util/delay.h> // delay functions

// array to store bit patterns for each digit
// low (0) turns on the led segment
const uint8_t digit_map[] = {
    0b00000000,  // 0
    0b11100100,  // 1
    0b10010000,  // 2
    0b11000000,  // 3
    0b01100100,  // 4
    0b01001000,  // 5
    0b00001000,  // 6
    0b11100000,  // 7
    0b00000000,  // 8
    0b01000000   // 9
};

// defining time variables
volatile uint8_t hours = 5, minutes = 11, seconds = 10;

// array to store display digits
uint8_t digits[6];
// function to calculate digit values for display
void update_digits() {
    digits[0] = hours / 10;   // first display
    digits[1] = hours % 10;   // second display
    digits[2] = minutes / 10; // third display
    digits[3] = minutes % 10; // fourth display
    digits[4] = seconds / 10; // fifth display
    digits[5] = seconds % 10; // sixth display
}

// function to update time at every second
void update_time() {
    seconds++;
    if (seconds >= 60) { seconds = 0; minutes++; }
    if (minutes >= 60) { minutes = 0; hours++; }
    if (hours >= 24) { hours = 0; } // 24-hour clock

    // update digits array after time update
    update_digits();
}

// interrupt service routine (ISR) for Timer1 compare match A
// calls update_time() function every second
ISR(TIMER1_COMPA_vect) {
    update_time();
}

// function to display a single digit (multiplexing)
void display_digit(uint8_t display, uint8_t digit) {
    PORTB &= ~(0b00011110); // turn off PB1-PB4
    PORTC &= ~(0b00000011); // Turn off PC0-PC1

    PORTD = digit_map[digit];  // set segments a-f
     // control segment G (PB0)
    if (digit == 0 || digit == 1 || digit == 7) {
        PORTB |= (1 << PB0); // turn on g segment
    } else {
        PORTB &= ~(1 << PB0); // turn off g segment
    }

    // enable the correct digit select pin
    if (display < 4) {
        // PB1-PB4 for first to fourth displays
        PORTB |= (1 << (display + 1));
    } else {
        // PC0-PC1 for fifth and sixth displays
        PORTC |= (1 << (display - 4));
    }

    // ensuring the digit is viible before switching
    _delay_ms(2);
}

int main(void) {
    // setting PD2-PD7 as output for segments a-f
    DDRD |= 0b11111100;
    // setting PB0 as output for segment g
    DDRB |= (1 << PB0);
    // setting PORTB (PIN 9-12) and PORTC (A0-A1) as output 
    // (for digit selection)
    DDRB |= (1 << PB1) | (1 << PB2) | (1 << PB3) | (1 << PB4);
    DDRC |= (1 << PC0) | (1 << PC1);

    // initialize display digits before Timer starts
    update_digits();
      // set-up Timer1 (1 hz interrupt)
    TCCR1B |= (1 << WGM12) | (1 << CS12) | (1 << CS10);
    OCR1A = 15625;  // compare value for 1-second tick
    TIMSK1 |= (1 << OCIE1A);  // enable Timer1 compare interrupt
    sei();  // enable global interrupts

    while (1) {
        for (uint8_t i = 0; i < 6; i++) {
            // cycle through all 6 digits
            display_digit(i, digits[i]);
        }
    }
}
\end{lstlisting}

\section{Code Explanation}
\begin{enumerate}
    \item The program sets the CPU frequency to \textbf{16 MHz} and includes necessary AVR libraries for \textbf{I/O operations, interrupts, and delays}.
    \item A \textbf{digit map array} is defined to store bit patterns for displaying numbers (0-9) on a \textbf{7-segment display}.
    \item Variables \texttt{hours}, \texttt{minutes}, and \texttt{seconds} store the current time, while an array \texttt{digits[6]} holds individual digits for display.
    \item The function \texttt{update\_digits()} extracts \textbf{each digit} from \texttt{hours}, \texttt{minutes}, and \texttt{seconds} to update the display.
    \item The function \texttt{update\_time()} increments the \textbf{time every second}, resetting values when they reach \textbf{60 (minutes/seconds) or 24 (hours)}.
    \item \textbf{Timer1 is configured} to generate an interrupt \textbf{every second}, calling \texttt{update\_time()} automatically.
    \item The \textbf{display uses multiplexing}, where only \textbf{one digit is turned on at a time} using \texttt{display\_digit()}, reducing the number of required pins.
    \item \textbf{PORTB, PORTC, and PORTD} are set as \textbf{output pins} to control the \textbf{segments and digit selection} for the display.
    \item The \textbf{main loop continuously refreshes} the 7-segment display by cycling through all \textbf{6 digits rapidly}.
    \item The system runs indefinitely, updating time \textbf{every second} and displaying it using \textbf{time-sharing multiplexing}.
\end{enumerate}

\section{Results}
\begin{itemize}
\item The clock successfully keeps time, updating every second.
\item The multiplexed display correctly shows the HH:MM:SS format.
\item Timer1 ISR ensures accurate timekeeping with minimal drift.
\end{itemize}


\section{Conclusion}
The experiment successfully demonstrates a working digital clock using an Arduino board and seven-segment displays. By leveraging timer interrupts and display multiplexing, an efficient and accurate timekeeping system is achieved.\\

Acknowledgement: code sourced from M srujana, EE24BTECH11060
\end{document}




